\ifoot{\href{https://missboulette.wordpress.com/2010/07/26/4-schritte-zum-ideal-gerosteten-sesam/}{https://missboulette.wordpress.com/2010/07/26/4-schritte-zum-ideal-gerosteten-sesam/}}
\section[4 Schritte zum ideal gerösteten Sesam]{\leafright\, 4 Schritte zum ideal gerösteten Sesam \leafleft}
\label{sec:4 Schritte zum ideal gerösteten Sesam}

\begin{minipage}[t]{0.34\textwidth}
\vspace{0pt}\fbox{\includegraphics[width=1\linewidth]{./Bilder/ideal_geroesteter_Sesam.png}}
\vspace{0.5cm}
\end{minipage}
\hfill
\begin{minipage}[t]{0.58\textwidth}
\vspace{0pt}
\subsection*{Zubereitung}
\begin{enumerate}[leftmargin=*, itemindent=14pt]

\item Dafür nehme ich 1-2 Tassen Sesam als Portion, kurz waschen, auf einem Haarsieb (falls nicht vorhanden ein Tuch zur Hilfe nehmen) gut abtropfen lassen.

\item Den nassen Sesam in einem Topf auf mittlerer Hitze unter ständigem Rühren mit einem breiten Spatel ohne Fett langsam rösten. Erst verdunstet das Wasser und es entsteht Dampf, aber dann erhitzen sich die Körner langsam und es fängt an zu knistern und zu knacken. Ständig weiter rühren nicht vergessen!! Nach einigen Minuten riecht es schon langsam nussig und aromatisch, die vormals hellen und platten Körner sind nun golden und bauchig, und es raucht sehr dezent. Ab diesem Zeitpunkt ist Multitasking angesagt!

\item Während man mit der einen Hand wie eine Maschine gleichmäßig weiterrührt, probiert man mit der anderen immer wieder und zerreibt dabei einige Sesamkörner zwischen Daumen und Zeigefinger. Erst wenn sich die Körner gut zerreiben lassen und die Finger dabei trocken bleiben, haben sie die richtige Röststufe und damit das volle Aroma erreicht.

\item Nun sofort in eine vorher bereitgestellte Schale (mise en place) umschütten und abkühlen lassen. Es geht hier um Sekunden! Falls ihr erst jetzt eine Schüssel suchen oder umständlich aus dem hintersten Regal herausholen müsstet, wäre es der Tod für die Sesamkörner. Noch schlimmer wäre es für sie im noch heißen Topf zu verweilen. Die ideale Röststufe liegt leider haarscharf neben der ranzigen. Sobald der Zenit erreicht ist fällt die Qualität rasant ab. Erkennen könnt ihr es an übermäßiger Rauchentwicklung, d.h. Fett tritt bereits aus den Körnern aus und verbrennt. Die Körner sind ölig, einige bereits geplatzt. Diese Körner würden nur wenige Tage gut schmecken, danach schnell unangenehm ranzig.
\end{enumerate}

Verpasst ihr den idealen Zeitpunkt dagegen nicht, kann der so geröstete Sesam fast einige Monate gut überstehen. Trocken, luftdicht und dunkel lagern.


Nach Belieben mit Currypulver bestäubt servieren.
\end{minipage}
\vfill
\decothreeright \, \textbf{Arbeitszeit:} wenige Minuten	 / \textbf{Schwierigkeitsgrad:} simpel	 / \decothreeleft \hfill Bewertung: \Circle  \Circle \Circle  \Circle \Circle
\ifoot{\href{https://www.chefkoch.de/rezepte/2133281343053838/Rinderrouladen-klassisch.html}{https://www.chefkoch.de/rezepte/2133281343053838/Rinderrouladen-klassisch.html}}
\section[Rinderrouladen klassisch]{\leafright\, Rinderrouladen klassisch \leafleft}
\begin{minipage}[t]{0.34\textwidth}
\vspace{0pt}
\fbox{\includegraphics[width=1\linewidth]{./Bilder/rinderrouladen-klassisch.png}}
\vspace{0.5cm}

\begin{small}
\begin{tabular}{R{1.6cm} L{3.8cm} }
\multicolumn{2}{c}{\textbf{Zutaten für 2 Portionen }}\\ \toprule

8 & 	Roulade(n) vom Rind \\ \midrule[0.1mm]
5  &	Zwiebel(n) \\ \midrule[0.1mm]
4  &	Gewürzgurke(n) \\ \midrule[0.1mm]
12 &Scheibe/n 	Frühstücksspeck \\ \midrule[0.1mm]
4 EL& 	Senf, mittelscharfer \\ \midrule[0.1mm]
1 Stück(e) &	Knollensellerie \\ \midrule[0.1mm]
1  	& Möhre(n) \\ \midrule[0.1mm]
1/2 Stange/n &	Lauch \\ \midrule[0.1mm]
1/2 Flasche &	Rotwein, guter \\ \midrule[0.1mm]
  	& Salz \\ \midrule[0.1mm]
  	& Pfeffer \\ \midrule[0.1mm]
1/2 Liter &	Rinderfond, kräftiger \\ \midrule[0.1mm]
 TL &	Speisestärke \\ \midrule[0.1mm]
1 Schuss &	Gurkenflüssigkeit \\ \midrule[0.1mm]
2 EL &	Butterschmalz  \\ \bottomrule

\end{tabular}
\end{small}
\end{minipage}
\hfill
\begin{minipage}[t]{0.58\textwidth}
\vspace{0pt}
\subsection*{Zubereitung}
\begin{enumerate}[leftmargin=*, itemindent=14pt]
\item Die Rinderrouladen aufrollen, waschen und mit Küchenkrepp trockentupfen. Zwiebeln in Halbmonde, Gurken in Längsstreifen schneiden, Schere und Küchengarn bereitstellen. 

\item Die ausgebreiteten Rouladen dünn mit Senf bestreichen, salzen und pfeffern, auf jede Roulade mittig in der Länge ca. 1/2 Zwiebel und 1 1/2 Scheiben Frühstücksspeck sowie 1/2 (evtl. mehr) Gurke verteilen. Nun von beiden Längsseiten etwas einschlagen, dann aufrollen und mit dem Küchengarn wie ein Postpaket verschnüren.

\item In einer Pfanne das Butterschmalz heiß werden lassen und die Rouladen dann rundherum darin anbraten, herausnehmen und in einen Schmortopf umfüllen.

\item Den Sellerie, die restliche Zwiebel, das Lauch und die Möhren kleinschneiden und in der Pfanne anbraten. Sobald sie halbwegs "blond" sind, kurz rühren, eine sehr dünne Schicht vom Rotwein angießen, nicht mehr rühren und die Flüssigkeit verdampfen lassen. Sobald das Gemüse dann wieder trockenbrät, wieder eine Schicht angießen, kurz rühren und weiter verdampfen lassen. Dies wiederholen, bis die 1/2 Flasche Wein aufgebraucht ist. Auf diese Art wird das Röstgemüse sehr braun (gut für den Geschmack und die Farbe der Soße) aber nicht trocken. Am Schluss mit der Fleischbrühe, etwas Salz und Pfeffer und einem guten Schuss Gurkensud auffüllen und dann in den Schmortopf zu den Rouladen geben. Den Topf entweder auf kleiner Flamme oder bei ca. 160 Grad im Backofen für 1 1/2 Stunden schmoren lassen. Ab und zu evtl. etwas Flüssigkeit zugießen.

\item Nach 1 1/2 Stunden testen, ob die Rouladen weich sind (einfach mal mit den Kochlöffel ein bisschen draufdrücken, sie sollten sich willig eindrücken lassen, wenn nicht, nochmal eine halbe Stunde weiterschmoren) und dann vorsichtig aus dem Topf heben, warm stellen.

\item Die Soße durch ein Sieb geben, aufkochen. Ca. 1 El Senf mit etwas Wasser und der Speisestärke gut verrühren und in die kochende Soße nach und nach unter Rühren eingießen, bis die gewünschte Konsistenz erreicht ist. Die Soße evtl. nochmal mit Salz, Pfeffer, Rotwein, Gurkensud abschmecken. 

\end{enumerate}
\end{minipage}
\vfill
\decothreeright \, \textbf{Arbeitszeit:} ca. 1 Std. / \textbf{Koch-Backzeit:} ca. 2 Std. /\textbf{Schwierigkeitsgrad:} pfiffig \decothreeleft \hfill \\ Bewertung:  \Circle  \Circle \Circle \Circle \Circle
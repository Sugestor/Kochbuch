\ifoot{\href{http://www.chefkoch.de/rezepte/1651531272966946/Schaschlik-wie-im-Kaukasus-grillen.html}{http://www.chefkoch.de/rezepte/Schaschlik-wie-im-Kaukasus-grillen.html}}
%\chapter{...alles...}
\section[Schaschlik wie im Kaukasus]{\leafright\, Schaschlik wie im Kaukasus \leafleft}
\begin{minipage}[t]{0.34\textwidth}
\vspace{0pt}
\fbox{\includegraphics[width=1\linewidth]{./Bilder/schaschlik-wie-im-kaukasus.png}}
\vspace{0.5cm}

\begin{small}
\begin{tabular}{R{1.6cm} L{3.8cm} }
\multicolumn{2}{c}{\textbf{Zutaten für 4 Portionen }}\\ \toprule
1 kg&	 Fleisch (Nackenfleisch von Schwein, Rind oder Lamm), frisch\\ \midrule[0.1mm]
3 &	 Gemüsezwiebel(n)\\ \midrule[0.1mm]
500 ml&	 Milch\\ \midrule[0.1mm]
1 Schuss&	 Essig\\ \midrule[0.1mm]
 	 &Salz und Pfeffer aus der Mühle\\ \midrule[0.1mm]
etwas&	 Tomatenmark\\ \midrule[0.1mm]
1 	& Kiwi\\ \bottomrule

\end{tabular}
\end{small}

\end{minipage}
\hfill
\begin{minipage}[t]{0.58\textwidth}
\vspace{0pt}
\subsection*{Zubereitung}
\begin{enumerate}[leftmargin=*, itemindent=14pt]
\item Den Schweinenacken in nicht zu kleine Würfel schneiden. Die Zwiebeln halbieren und in halbe Ringe schneiden. 

\item Tomatenmark, Salz, Pfeffer und einen Spritzer Essig mischen und mit der Hand schön in das Fleisch einmassieren. Alles in eine große Schüssel geben und mit Milch auffüllen. Zugedeckt über Nacht an einem kühlen Ort einziehen lassen.

\item Am nächsten Tag die Marinade probieren und evtl. mit Salz und Pfeffer nachwürzen. Ca. 2 Stunden vorm Grillen die Kiwi schälen und in kleine Stücke schneiden. In die Marinade geben und ebenfalls in das Fleisch einmassieren. Bitte die Kiwi nicht zu lange (max. 2 Stunden) in der Marinade lassen, da sonst das Fleisch zu weich wird und vom Spieß fällt.

\item Dann das Fleisch auf Spieße ziehen und grillen. Kurz vor dem Verzehr mit Essigwasser beträufeln. 
\end{enumerate}
Sehr lecker schmecken auch frische Zwiebelringe in leichtes Essigwasser eingelegt und zum Schaschlik serviert.
\end{minipage}
\vfill
\decothreeright \, \textbf{Arbeitszeit:} ca. 20 Min. / \textbf{Schwierigkeitsgrad:} normal \decothreeleft \hfill Bewertung: \CIRCLE \CIRCLE \CIRCLE \CIRCLE \LEFTcircle 
\ifoot{\href{http://www.chefkoch.de/rezepte/1780071287904514/Bolognese-auf-meine-Art.html}{http://www.chefkoch.de/rezepte/1780071287904514/Bolognese-auf-meine-Art.html}}
\section[Bolognese]{\leafright\, Bolognese \leafleft}
\begin{minipage}[t]{0.34\textwidth}
\vspace{0pt}\fbox{\includegraphics[width=1\linewidth]{./Bilder/bolognese.png}}
\vspace{0.5cm}

\begin{small}
\begin{tabular}{R{1.6cm} L{3.8cm}  }
\multicolumn{2}{c}{\textbf{Zutaten für 4 Portionen }}\\ \toprule
500 g& Hackfleisch \\ \midrule
1 Becher&	 Sahne\\ \midrule
300 g&	 Champignons\\ \midrule
2 &	 Zwiebeln\\ \midrule
1 EL&	 Kreuzkümmel\\ \midrule
je 1 TL& Oregano, Majoran, Thymian, Basilikum\\ \midrule
3 &	 Lorbeerblätter\\ \midrule
1 Stück& Schokolade mit 80\% Kakaoanteil\\ \midrule
1 EL&	 Essig\\ \midrule
1 EL&	 Salz und Pfeffer\\ \midrule
1 Pck.&	 Tomate(n), passierte\\ \midrule
30 g&	 Tomatenmark\\ \midrule
2 &	 Knoblauchzehe(n)\\ \bottomrule
\end{tabular}
\end{small}

\end{minipage}
\hfill
\begin{minipage}[t]{0.58\textwidth}
\vspace{0pt}
\subsection*{Zubereitung}
\begin{enumerate}[leftmargin=*, itemindent=14pt]
\item Die Zwiebeln fein würfeln. Das Hackfleisch in eine heiße Pfanne geben, mit Salz und Pfeffer würzen und mit den Zwiebeln zusammen anbraten. Die Pilze in dicke Scheiben (5 mm) schneiden, dazugeben und alles kräftig anbraten.\\

\item Mit Sahne ablöschen, Knoblauch fein hacken und hinzugeben. Die Röstaromen vom Pfannenboden lösen und jetzt alle Kräuter und Gewürze außer dem Zucker, der Schokolade und dem Essig hinzufügen.\\

\item Auf mittlere Hitze herunter stellen und 1-2 Minuten kochen lassen. Dann die passierten Tomaten und das Tomatenmark, sowie Zucker und Essig hinzufügen und die Soße auf mittlerer Hitze kochen lassen, bis sie schön dickflüssig geworden ist. Vorsicht, wenn die Soße dicker wird, brennt sie leicht an, also häufig umrühren!\\

\item Die Soße ist fertig, wenn sich oben ein Fettspiegel gebildet hat und die Farbe schön rötlich geworden ist. Wenn sie noch zu hell ist, etwas Wasser hinzugeben und weiter kochen, bis sich der Fettspiegel und die Farbe gebildet haben. \\

\item Jetzt die Schokolade hinzufügen, schmelzen lassen und verrühren. Am Ende nochmal abschmecken. \\
\end{enumerate}
Tipp: Je öfter man nochmal Wasser hinzufügt und die Soße einkochen lässt, um so besser schmeckt sie, das darf gerne auch 2 Stunden dauern.\\

Am besten schmecken mir persönlich dazu al dente gekochte Linguine bzw. Bavette.
\end{minipage}
\vfill
\decothreeright \, \textbf{Arbeitszeit:} ca. 30 Min.	 / \textbf{Schwierigkeitsgrad:} normal	 / \decothreeleft \hfill Bewertung: \CIRCLE  \CIRCLE \CIRCLE \CIRCLE \LEFTcircle 
\ifoot{\href{http://www.chefkoch.de/rezepte/376181123622498/Ungewoehnlicher-Rote-Bete-Salat.html}{http://www.chefkoch.de/rezepte/376181123622498/Ungewoehnlicher-Rote-Bete-Salat.html}}
\section[Rote-Bete-Salat mit Erdnussbutter]{\leafright\, Rote-Bete-Salat mit Erdnussbutter \,\leafleft}
\begin{minipage}[t]{0.34\textwidth}
\vspace{0pt}
\fbox{\includegraphics[width=1\linewidth]{./Bilder/rote-bete-salat.png}}
\vspace{0.5cm}

\begin{small}
\begin{tabular}{R{1.6cm} L{3.8cm} }
\multicolumn{2}{c}{\textbf{Zutaten für 4 Portionen }}\\ \toprule
4 Knollen&	 Rote Bete (oder ein Glas)\\ \midrule[0.1mm]
2 Zehen&	 Knoblauch\\ \midrule[0.1mm]
2 Becher&	 Joghurt, klein\\ \midrule[0.1mm]
2 EL&	 Erdnussbutter\\ \midrule[0.1mm]
 	& Pfeffer, schwarz, frisch\\ \bottomrule
\end{tabular}
\end{small}
\end{minipage}
\hfill
\begin{minipage}[t]{0.58\textwidth}
\vspace{0pt}
\subsection*{Zubereitung}
\begin{enumerate}[leftmargin=*, itemindent=14pt]
\item Den Joghurt mit der Erdnussbutter, dem durchgepressten Knoblauch und frisch gemahlenem schwarzen Pfeffer mischen, über die in Scheiben geschnittene Rote Bete geben.
\end{enumerate}
\end{minipage}
\vfill
\decothreeright \, \textbf{Arbeitszeit:} 10 Min. / \textbf{Schwierigkeitsgrad:} simpel \decothreeleft \hfill Bewertung:  \Circle \Circle \Circle \Circle \Circle
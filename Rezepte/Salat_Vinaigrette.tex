\ifoot{\href{http://www.cookingforengineers.com/recipe/44/Basic-Vinaigrette-Salad-Dressing}{http://www.cookingforengineers.com/recipe/44/Basic-Vinaigrette-Salad-Dressing}}
\section[Basic Vinaigrette Salat Dressing]{\leafright\, Basic Vinaigrette Salat Dressing \leafleft}
\begin{minipage}[t]{0.34\textwidth}
\vspace{0pt} %\fbox{\includegraphics[width=1\linewidth]{./Bilder/rezepte_gefuellte-paprika.png}}
\vspace{0.5cm}

\begin{small}
\begin{tabular}{R{1.7 cm} L{3.7cm} }
\multicolumn{2}{c}{\textbf{Zutaten für 1 Portionen}}\\ \toprule
1  Tasse& extra natives Olivenöl\\ \midrule[0.1mm]
1/2 Becher & Balsamico Essig\\ \midrule[0.1mm]
3 & Knoblauchzehen (gepresst)\\ \midrule[0.1mm]
1 Tl. & Oregano\\ \midrule[0.1mm]
1/4 Tl. & Rosmarin, zerdrückt\\ \bottomrule
\end{tabular}
\end{small}
\end{minipage}
\hfill
\begin{minipage}[t]{0.58\textwidth}
\vspace{0pt}
\section*{Zubereitung}
\begin{enumerate}[leftmargin=*, itemindent=14pt]
\item Aus Hackfleisch, Reis, Ei, Zwiebel und Knoblauch einen Hackfleischteig herstellen und mit den Gewürzen abschmecken. In die Paprikaschoten füllen und aus dem Rest Hackfleischbällchen formen.
\item Die Butter in einem Topf schmelzen, das Mehl dazugeben und etwas anrösten. Mit Gemüsebrühe ablöschen, das Tomatenmark dazugeben und aufkochen lassen. Mit Salz, Pfeffer und etwas Zucker abschmecken. 
\item Die gefüllten Paprikaschoten und die Bällchen in die Soße geben und entweder auf dem Herd oder im Backofen 30-40 min schmoren lassen.
\item Dazu gibt es Reis. 
\end{enumerate}
Alternativ kann man auch Tomatenpüree statt des Tomatenmarks verwenden. Dann etwas weniger Brühe verwenden.
\end{minipage}
\vfill
\decothreeright \, \textbf{Arbeitszeit:} ca. 40 Min.	 / \textbf{Schwierigkeitsgrad:} normal	 / \decothreeleft \hfill Bewertung: \Circle  \Circle \Circle  \Circle \Circle
\ifoot{\href{http://www.chefkoch.de/rezepte/237711096640386/Blumenkohl-gebraten.html}{chefkoch.de/Blumenkohl-gebraten} \& \href{http://www.chefkoch.de/rezepte/1686711276938595/Blumenkohl-im-Bierteig-ausgebacken.html}{chefkoch.de/Blumenkohl-im-Bierteig-ausgebacken}}
\section[Blumenkohl gebraten mit Bechamelsoße]{\leafright\, Blumenkohl gebraten mit Bechamelsoße \,\leafleft}
\begin{minipage}[t]{0.34\textwidth}
\vspace{0pt}
\fbox{\includegraphics[width=1\linewidth]{./Bilder/blumenkohl_gebraten.png}}
\vspace{0.5cm}

\begin{small}
\begin{tabular}{R{1.2cm} L{4.2cm} }
\multicolumn{2}{c}{\textbf{Zutaten für 4 Portionen }}\\ \toprule
1 &	 Blumenkohl\\ \midrule[0.1mm]
3 &	 Ei(er)\\ \midrule[0.1mm]
 &	 Semmel(n), geriebene Paniermehl\\ \midrule[0.1mm]
 &	 Salz, Pfeffer, Muskat, Worcestershiresauce\\ \midrule[0.1mm]
 &	 Butter  Öl\\ \midrule[0.2mm]\addlinespace
\multicolumn{2}{c}{Alternative: Bierteig}\\  \midrule[0.1mm] 
125 ml& Bier, hell\\  \midrule[0.1mm]
2& Eier\\  \midrule[0.1mm]
140 g& Mehl\\ \midrule[0.1mm]
& Salz \\ \midrule[0.1mm]
1 Tl.& Öl \\ \bottomrule
\end{tabular}
\end{small}
\end{minipage}
\hfill
\begin{minipage}[t]{0.58\textwidth}
\vspace{0pt}
\subsection*{Zubereitung}
\begin{enumerate}[leftmargin=*, itemindent=14pt]
\item Den Blumenkohl wie gewohnt in kleine Röschen teilen und bissfest in Salzwasser kochen. Kochwasser aufheben. Soweit abkühlen lassen dass man diesen zum panieren in den Händen halten kann ohne sich zu verbrennen. 

\item Panieren mit Ei und Paniermehl (wie ein Schnitzel), vielleicht das Ei vorher noch mit Pfeffer würzen. Anschließend in der Pfanne Butter erhitzen und den Blumenkohl schön goldbraun ringsherum braten.

\item Ein großes Stück Margarine im Topf erhitzen und das Mehl darin anschwitzen. Anschließend unter Rühren das Blumenkohlwasser hinzufügen. Dann soviel Milch hinzufügen bis sich eine sämige, nicht zu feste Soße gebildet hat. In die Soße kommt noch etwas Zitronensaft, Muskat und Worchester-Soße. Mit Salz abschmecken und fertig. 
\end{enumerate}

\subsection*{Alternative: Bierteig}
\begin{enumerate}[leftmargin=*, itemindent=14pt]
\item Für den Bierteig die Eiweiße steif schlagen, dann in den Kühlschrank stellen.

\item Mehl, Bier, Öl, Salz, Muskatnuss und Eigelb schnell miteinander zu einem geschmeidigen Teig verrühren, nicht zu lange rühren, sonst wird der Teig später zäh. Den Teig etwas rasten lassen, dann zügig, aber vorsichtig! den Eischnee unterheben.

\item Die Blumenkohlröschen durch den Teig ziehen und im heißen Öl schwimmend ausbacken, bis sie schön goldgelb sind.
\end{enumerate}
\end{minipage}
\vfill
\decothreeright \, \textbf{Arbeitszeit:} 30 Min. / \textbf{Schwierigkeitsgrad:} simpel \decothreeleft \hfill Bewertung:  \CIRCLE \CIRCLE \CIRCLE \CIRCLE  \Circle
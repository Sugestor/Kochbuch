\ifoot{\href{http://www.chefkoch.de/rezepte/1621901269418083/Fladenbrotburger.html}{http://www.chefkoch.de/rezepte/1621901269418083/Fladenbrotburger.html}}
\section[Fladenbrotburger]{\leafright\, Fladenbrotburger \leafleft}
\begin{minipage}[t]{0.34\textwidth}
\vspace{0pt}\fbox{\includegraphics[width=1\linewidth]{./Bilder/fladenbrotburger.png}}
\vspace{0.5cm}

\begin{small}
\begin{tabular}{R{1.6cm} L{3.8cm} }
\multicolumn{2}{c}{\textbf{Zutaten für 4 Portionen}}\\ \toprule
1 &	 Fladenbrot\\ \midrule[0.1mm]
500 g&	 Hackfleisch, gemischt\\ \midrule[0.1mm]
3 große&	 Tomaten\\ \midrule[0.1mm]
1 Pck.&	 Schafskäse\\ \midrule[0.1mm]
100 g&	 Crème fraîche\\ \midrule[0.1mm]
 	& Salz und Pfeffer\\ \midrule[0.1mm]
 	& Paprikapulver\\ \midrule[0.1mm]
 	& Thymian, getrocknet\\ \bottomrule
\end{tabular}
\end{small}

\end{minipage}
\hfill
\begin{minipage}[t]{0.58\textwidth}
\vspace{0pt}
\subsection*{Zubereitung}
\begin{enumerate}[leftmargin=*, itemindent=14pt]
\item Hackfleisch in großer Pfanne braten und zerteilen, bis eine schöne Bräune erreicht ist. Mit Salz, Pfeffer, Paprikapulver und Thymian würzen.\\

\item Fladenbrot längs aufschneiden, das Hackfleisch gleichmäßig auf der Unterseite verteilen, einen kleinen Rand von 0,5 cm lassen. Nun mit Tomatenscheiben und dem aufgeschnittenen Schafskäse belegen, Crème fraîche in Klecksen darauf geben und die obere Brothälfte auflegen. Auf ein Backblech legen und bei 160 Grad Grad Umluft 12-15 min aufbacken.\\
\end{enumerate}
Lecker auch mit Mozzarella!
\end{minipage}
\vfill
\decothreeright \, \textbf{Arbeitszeit:} ca. 15 Min.	 / \textbf{Schwierigkeitsgrad:} simpel	 / \decothreeleft \hfill Bewertung: \Circle \Circle \Circle  \Circle \Circle
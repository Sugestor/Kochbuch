\ifoot{\href{http://www.chefkoch.de/rezepte/293201108478436/Afrikanische-Erdnusssauce.html}{http://www.chefkoch.de/rezepte/293201108478436/Afrikanische-Erdnusssauce.html}}
\section[Afrikanische Erdnusssauce]{\leafright\, Afrikanische Erdnusssauce \leafleft}
\begin{minipage}[t]{0.34\textwidth}
\vspace{0pt}\fbox{\includegraphics[width=1\linewidth]{./Bilder/rezepte_afrikanische-erdnusssauce.png}}
\vspace{0.5cm}

\begin{small}
\begin{tabular}{R{1.6 cm} L{3.8cm} }
\multicolumn{2}{c}{\textbf{Zutaten für 2 Portionen}}\\ \toprule
2 EL&	 Öl\\ \midrule[0.1mm]
1& 	 Zwiebel, fein gehackt\\ \midrule[0.1mm]
1 kl. Dose&	 Tomatenmark\\ \midrule[0.1mm]
3/4 Liter&	 Wasser\\ \midrule[0.1mm]
1 bis 4 TL&	 Sambal Oelek\\ \midrule[0.1mm]
200 g&	 Erdnusscreme (Pindakaas)\\ \midrule[0.1mm]
 	& Salz\\ \midrule[0.5mm]
 & Couscous oder Reis\\ \midrule[0.1mm]
 & Hähnchenbrust oder Rindfleisch\\ \bottomrule
\end{tabular}
\end{small}
\end{minipage}
\hfill
\begin{minipage}[t]{0.58\textwidth}
\vspace{0pt}
\subsection*{Zubereitung}
\begin{enumerate}[leftmargin=*, itemindent=14pt]
\item Öl in einem Topf erhitzen, gehackte Zwiebel dazugeben, glasig braten lassen. Mit Wasser ablöschen. Salz, Sambal Olek und Tomatenmark unterrühren. Kurz aufkochen lassen. Von der Feuerstelle nehmen. Erdnusscreme einrühren und auf die Feuerstelle geben. Unter Rühren aufkochen lassen.\\

\leafNE\, oder
\item \begin{enumerate}[leftmargin=*, itemindent=18pt]

\item Das Fleisch (traditionell Rindfleisch) klein schneiden und zusammen mit den Zwiebeln in einem hohen Topf anbraten. Danach die oben angegebenen Zutaten dazugeben und auf nicht zu starker Flamme kochen lassen. Gelegentlich umrühren und mindestens 40 Minuten köcheln lassen (damit das Fleisch richtig durchgezogen ist mit dem Erdnussgeschmack). Vorsicht mit dem Wasser! Nur soviel dazu geben, dass die Sauce sämig aber nicht zu flüssig wird. 

\item Diese Sauce wird normalerweise sehr scharf gegessen, also seid grosszügig mit dem Cayenne-Pfeffer.

Noch ein Tipp: Es ist manchmal schwierig Pindakaas (Erknusscreme) in Deutschland zu bekommen - alternativ kann man Erdnussbutter nehmen (die findet ihr im "Nutella-Regal")\\
Um dem ganzen etwas mehr Rafinesse zu geben, serviert es doch mal mit Kokosnussreis - Wasser und Kokosnussmilch halb und halb mischen und den Reis dazugeben. 

\end{enumerate}
\end{enumerate}
\end{minipage}
\vfill
\decothreeright \, \textbf{Arbeitszeit:} ca. 10 Min.	 / \textbf{Schwierigkeitsgrad:} simpel	 / \decothreeleft \hfill Bewertung: \CIRCLE \CIRCLE \LEFTcircle  \Circle \Circle
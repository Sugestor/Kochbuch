\documentclass[
	a4paper, % Papierformat
	BCOR=20mm,
	ngerman, % für Umlaute, Silbentrennung etc.
	titlepage, % es wird eine Titelseite verwendet
	parskip=half, % Abstand zwischen Absätzen (halbe Zeile)
	final % Status des Dokuments (final/draft)
]{scrreprt}
\usepackage{scrhack}

%\overfullrule=5pt %visualisiere volle/leere Boxen

\usepackage{lmodern} % bessere Fonts
\usepackage[T1]{fontenc} %Vektorschriften
\usepackage[utf8]{inputenc} %üäöß usw richtg erkennen...
\usepackage[ngerman]{babel} %english,french....
\usepackage[babel]{microtype} %verbessertes schriftbild
\usepackage{relsize} % Schriftgröße relativ festlegen
\usepackage{subfig,wrapfig}
% zum Einbinden von Programmcode

\usepackage{kpfonts} %http://sunsite.informatik.rwth-aachen.de/ftp/pub/mirror/ctan/fonts/kpfonts/doc/kpfonts.pdf
%\usepackage{mathptmx} % für Schrift mit Serifen
%\usepackage[scaled=.90]{helvet} % für Schrift ohne Serifen
%\usepackage{courier} % für Schrift mit konstanter Breite


\usepackage{calc}% http://ctan.org/pkg/calc
\usepackage{enumitem} %erweiterte Listen

\usepackage{fourier-orns} %schnörkel
\usepackage{wasysym} %erweiterte Symbole

\usepackage{lastpage}

% -----------------------------------------------
\usepackage{listings}
\usepackage{xcolor}

\usepackage{graphicx} % um Bilder mit \graphics einbinden zu können
\usepackage[dvips]{epsfig} % um Bilder zu skalieren

\definecolor{darkblue}{rgb}{0,0,.5}
\definecolor{darkmagenta}{rgb}{.5,0,.6}
\usepackage[  %zum einbinden von urls
    bookmarks,
    bookmarksopen=true,
    colorlinks=true,% diese Farbdefinitionen zeichnen Links im PDF farblich aus
    linkcolor=darkblue, %red, % einfache interne Verknüpfungen
    anchorcolor=black,% Ankertext
    citecolor=darkmagenta, % Verweise auf Literaturverzeichniseinträge im Text
    filecolor=magenta, % Verknüpfungen, die lokale Dateien öffnen
    menucolor=red, % Acrobat-Menüpunkte
    urlcolor=blue,
% diese Farbdefinitionen sollten für den Druck verwendet werden (alles schwarz)
    %linkcolor=black, % einfache interne Verknüpfungen
    %anchorcolor=black, % Ankertext
    %citecolor=black, % Verweise auf Literaturverzeichniseinträge im Text
    %filecolor=black, % Verknüpfungen, die lokale Dateien öffnen
    %menucolor=black, % Acrobat-Menüpunkte
    %urlcolor=black,
    backref,
    plainpages=false, % zur korrekten Erstellung der Bookmarks
    pdfpagelabels, % zur korrekten Erstellung der Bookmarks
    hypertexnames=false, % zur korrekten Erstellung der Bookmarks
    linktocpage % Seitenzahlen anstatt Text im Inhaltsverzeichnis verlinken
]{hyperref}


%gute Tabellen
\usepackage{booktabs,colortbl,tabularx}
\usepackage{multirow}

% Einfache Definition der Zeilenabstände und Seitenränder etc. -----------------
\usepackage{setspace}
\usepackage{geometry}


\geometry{paper=a4paper,
left=22mm,
right=18mm,
top=30mm,
bottom=20mm}

%Tabelle feste spaltenbreite mit Ausrichtung rechts, center oder links
\usepackage{tabularx}
\newcolumntype{L}[1]{>{\raggedright\arraybackslash}p{#1}} % linksbündig mit Breitenangabe
\newcolumntype{C}[1]{>{\centering\arraybackslash}p{#1}} % zentriert mit Breitenangabe
\newcolumntype{R}[1]{>{\raggedleft\arraybackslash}p{#1}} % rechtsbündig mit Breitenangabe


% Kopf- und Fußzeilen ----------------------------------------------------------
\usepackage{scrlayer-scrpage}
\pagestyle{scrheadings}
% Kopf- und Fußzeile auch auf Kapitelanfangsseiten
% loescht voreingestellte Stile
\clearscrheadings
\clearscrplain

\renewcommand*{\chapterpagestyle}{scrheadings}
% Schriftform der Kopfzeile
\renewcommand{\headfont}{\normalfont}
\setlength{\headheight}{0mm} % Höhe der Kopfzeile

% Schriftform der Fußzeile
\newcommand*{\totalpagemark}{\thepage/\pageref{LastPage}}
\ofoot{\totalpagemark}
\setlength{\footskip}{12mm}

%Überschriften Anpassen
\renewcommand*{\othersectionlevelsformat}[3]{%
  \makebox[1.5cm][l]{#3\autodot}%
}%
\renewcommand*{\chapterformat}{%
  \makebox[1.5cm][l]{\thechapter\autodot}%
}
\renewcommand{\chapterheadstartvskip}{\vspace*{-4\topskip}}
\renewcommand{\chapterheadendvskip}{\vspace*{0.8\topskip}}

\setlength{\fboxsep}{1.7mm}

%%%%%%%%%%%%%%%%%%%%%%%%%%%%%%%%%%%%%%%%%%%%%%%
% Wir definieren einige Dinge nur einmal und verwenden sie dann mehrfach.
\newcommand*\Title{Mein Kochbuch}
\newcommand*\Subject{}
\newcommand*\Author{Alexander Schäfer}
\newcommand*\Keywords{}

% Wir wollen einen Index:
\usepackage{makeidx}\makeindex

% Wir wollen aktive Links und einige Dokumentinformationen:
\usepackage{hyperref}
\hypersetup{%
  pdftitle={\Title},
  pdfsubject={\Subject},
  pdfauthor={\Author},
  pdfkeywords={\Keywords}
}

\title{\Title\\}
\author{\Large \Author}
\begin{document}
%%%%%%%%%%%%%%%%%%%%%%%%
\maketitle  % erzeugt die Titelseite
\tableofcontents % erzeugt das Inhaltsverzeichnis
%\listoffigures % erzeugt das Abbildungsverzeichnis
%\listoftables % erzeugt das Abbildungsverzeichnis
%%%%%%%%%%%%%%%%%%%%%%%%
%\chapter{Fleischgerichte}
\ifoot{\href{http://www.chefkoch.de/rezepte/1832561297156839/Wildschweingulasch-mit-Waldpilzen.html}{http://www.chefkoch.de/rezepte/Wildschweingulasch-mit-Waldpilzen.html}}
\section[Wildschweingulasch mit Waldpilzen]{\leafright\, Wildschweingulasch mit Waldpilzen \,\leafleft}
\begin{minipage}[t]{0.34\textwidth}
\vspace{0pt}
\fbox{\includegraphics[width=1.0\textwidth]{./Bilder/wildschweingulasch-waldpilzen.png}}
\vspace{0.5cm}

\begin{small}
\begin{tabular}{R{1.7cm} L{3.7cm} }
\multicolumn{2}{c}{\textbf{Zutaten für 4 Portionen }}\\ \toprule
500 g&	 Gulasch vom Wildschwein\\ \midrule[0.1mm]
2 &	 Zwiebeln\\ \midrule[0.1mm]
2 EL&	 Tomatenmark\\ \midrule[0.1mm]
250 ml&	 Rotwein, trocken\\ \midrule[0.1mm]
400 ml&	 Wildfond\\ \midrule[0.1mm]
400 ml&	 Rinderbrühe, (Instant)\\ \midrule[0.1mm]
500 g&	 Pilze, gemischt, TK; auf gute Qualität achten\\ \midrule[0.1mm]
2 &	 Wacholderbeeren\\ \midrule[0.1mm]
1 	& Piment\\ \midrule[0.1mm]
2 	& Lorbeerblätter\\ \midrule[0.1mm]
3/4 Becher&	 Crème fraîche\\ \midrule[0.1mm]
 &	 Salz und Pfeffer\\ \bottomrule
\end{tabular}
\end{small}
\end{minipage}
\hfill
\begin{minipage}[t]{0.58\textwidth}
\vspace{0pt}
\subsection*{Zubereitung}
\begin{enumerate}[leftmargin=*, itemindent=14pt]
\item Das Gulasch waschen, trocken tupfen und mit Salz und Pfeffer würzen. Die Zwiebeln in feine Ringe schneiden.

\item Das Gulasch in etwas Öl oder Schmalz rundherum anbraten bis er eine schöne Farbe hat. Dann herausnehmen und abgedeckt zur Seite stellen. Die Zwiebelringe im gleichen Fett anbraten bis sie Farbe genommen haben. Die 2 Löffel Tomatenmark dazugeben und kurz mitbraten. Mit dem Rotwein ablöschen und aufkochen, den Wildfond angießen und aufkochen und dann mit der Brühe ergänzen. Wacholderbeeren, Lorbeerblätter und Piment dazugeben und mit geschlossem Deckel 1 Stunde und 45 Minuten schmoren lassen. Gelegentlich umrühren.

\item Ca. 15 Minuten bevor die Zeit um ist, die Waldpilze tiefgefroren in eine zweite Pfanne geben und auf großer Hitze braten bis das austretende Wasser vollständig verdampft ist. Mit Salz und Pfeffer würzen, mit etwas Wasser oder Brühe ablöschen und zu dem Wildschweingulasch geben.

\item Zusammen nochmal 10 Minuten weiterköcheln. Dann Creme fraiche einrühren, Lorbeerblätter, Wacholderbeeren und Piment entfernen und auf gewünschte Konsistenz eindicken. Falls notwendig (in der Regel nicht) mit Salz und Pfeffer würzen.
\end{enumerate}
Dazu passen Spätzle und evtl. Rotkraut.
\end{minipage}
\vfill
\decothreeright \, \textbf{Arbeitszeit:} 15 Min. / \textbf{Schwierigkeitsgrad:} normal \decothreeleft \hfill Bewertung:  \CIRCLE \CIRCLE \CIRCLE \CIRCLE \CIRCLE
\ifoot{\href{http://www.chefkoch.de/rezepte/1112181217260303/Lasagne-Bolognese.html}{http://www.chefkoch.de/rezepte/1112181217260303/Lasagne-Bolognese.html}}
\section[Lasagne Bolognese]{\leafright\, Lasagne Bolognese \leafleft}
\begin{minipage}[t]{0.34\textwidth}
\vspace{0pt}
\fbox{\includegraphics[width=1\linewidth]{./Bilder/lasagne-bolognese.png}}
\vspace{0.5cm}

\begin{small}
\begin{tabular}{R{1.6cm} L{3.8cm} }
\multicolumn{2}{c}{\textbf{Zutaten für 4 Portionen }}\\ \toprule
8 	& Lasagneplatte(n), gekocht\\ \midrule[0.1mm]
600 g&	 Hackfleisch\\ \midrule[0.1mm]
3 EL&	 Olivenöl\\ \midrule[0.1mm]
1 große	& Möhre(n)\\ \midrule[0.1mm]
1 Stück	& Sellerie\\ \midrule[0.1mm]
1 große	& Zwiebel(n)\\ \midrule[0.1mm]
2 Zehe/n&	 Knoblauch\\ \midrule[0.1mm]
200 ml	& Weißwein\\ \midrule[0.1mm]
500 ml&	 Brühe\\ \midrule[0.1mm]
3 EL	& Tomatenmark\\ \midrule[0.1mm]
 	& Salz und Pfeffer, Zucker\\ \midrule[0.1mm]
2 EL& Butter\\ \midrule[0.1mm]
3 EL& Mehl gestr.\\ \midrule[0.1mm]
500 ml&	 Milch\\ \midrule[0.1mm]
 	 &Muskat\\ \midrule[0.1mm]
100 g&	 Parmesan, frisch gerieben\\ \bottomrule
\end{tabular}
\end{small}
\end{minipage}
\hfill
\begin{minipage}[t]{0.58\textwidth}
\vspace{0pt}
\subsection*{Zubereitung}
\begin{enumerate}[leftmargin=*, itemindent=14pt]
\item Möhre, Sellerie, Zwiebel + Knoblauch putzen und in Würfel schneiden. Das Öl erhitzen, Würfel gut anbraten und wieder aus der Pfanne nehmen. 

\item Nun das Hackfleisch zur Hälfte zufügen und krümelig bzw. kross anbraten. Wieder aus der Pfanne nehmen und das restliche Hack anbraten. Das bereits gebratene Hack und Gemüse wieder zufügen, mit dem Wein ablöschen und fast verkochen lassen. Tomatenmark dazu geben, etwas angehen lassen und mit der Brühe auffüllen. Mit Deckel sämig einkochen lassen. Mit Salz, Pfeffer und einer Prise Zucker würzen.

\item Für die Bechamel die Butter in einem Topf zerlassen. Das Mehl einrühren und kurz anschwitzen, dann unter Rühren nach und nach die Milch zugießen. Mit Muskat und Salz abschmecken und einmal aufkochen lassen.

\item Auflaufform fetten und mit Lasagneblättern belegen. Einige Löffel Bechamel darauf verteilen, mit etwas Parmesan bestreuen und etwas von der Bolognese darüber geben. So weiter schichten, bis alle Zutaten verbraucht sind. Die oberste Schicht sollte aus der Bechamelsauce bestehen, die gleichmäßig mit dem Parmesan bestreut wird.

\item Im vorgeheizten Backofen bei 180°C ca. 20 Minuten überbacken. Bevor die Lasagne portioniert wird, einige Minuten ruhen lassen. Guten Appetit!
\end{enumerate}
\end{minipage}
\vfill
\decothreeright \, \textbf{Arbeitszeit:} ca. 1 1/4h / \textbf{Schwierigkeitsgrad:} normal \decothreeleft \hfill Bewertung:  \CIRCLE \CIRCLE \CIRCLE \CIRCLE \CIRCLE
\ifoot{\href{http://www.chefkoch.de/rezepte/1780071287904514/Bolognese-auf-meine-Art.html}{http://www.chefkoch.de/rezepte/1780071287904514/Bolognese-auf-meine-Art.html}}
\section[Bolognese]{\leafright\, Bolognese \leafleft}
\begin{minipage}[t]{0.34\textwidth}
\vspace{0pt}\fbox{\includegraphics[width=1\linewidth]{./Bilder/bolognese.png}}
\vspace{0.5cm}

\begin{small}
\begin{tabular}{R{1.6cm} L{3.8cm}  }
\multicolumn{2}{c}{\textbf{Zutaten für 4 Portionen }}\\ \toprule
500 g& Hackfleisch \\ \midrule
1 Becher&	 Sahne\\ \midrule
300 g&	 Champignons\\ \midrule
2 &	 Zwiebeln\\ \midrule
1 EL&	 Kreuzkümmel\\ \midrule
je 1 TL& Oregano, Majoran, Thymian, Basilikum\\ \midrule
3 &	 Lorbeerblätter\\ \midrule
1 Stück& Schokolade mit 80\% Kakaoanteil\\ \midrule
1 EL&	 Essig\\ \midrule
1 EL&	 Salz und Pfeffer\\ \midrule
1 Pck.&	 Tomate(n), passierte\\ \midrule
30 g&	 Tomatenmark\\ \midrule
2 &	 Knoblauchzehe(n)\\ \bottomrule
\end{tabular}
\end{small}

\end{minipage}
\hfill
\begin{minipage}[t]{0.58\textwidth}
\vspace{0pt}
\subsection*{Zubereitung}
\begin{enumerate}[leftmargin=*, itemindent=14pt]
\item Die Zwiebeln fein würfeln. Das Hackfleisch in eine heiße Pfanne geben, mit Salz und Pfeffer würzen und mit den Zwiebeln zusammen anbraten. Die Pilze in dicke Scheiben (5 mm) schneiden, dazugeben und alles kräftig anbraten.\\

\item Mit Sahne ablöschen, Knoblauch fein hacken und hinzugeben. Die Röstaromen vom Pfannenboden lösen und jetzt alle Kräuter und Gewürze außer dem Zucker, der Schokolade und dem Essig hinzufügen.\\

\item Auf mittlere Hitze herunter stellen und 1-2 Minuten kochen lassen. Dann die passierten Tomaten und das Tomatenmark, sowie Zucker und Essig hinzufügen und die Soße auf mittlerer Hitze kochen lassen, bis sie schön dickflüssig geworden ist. Vorsicht, wenn die Soße dicker wird, brennt sie leicht an, also häufig umrühren!\\

\item Die Soße ist fertig, wenn sich oben ein Fettspiegel gebildet hat und die Farbe schön rötlich geworden ist. Wenn sie noch zu hell ist, etwas Wasser hinzugeben und weiter kochen, bis sich der Fettspiegel und die Farbe gebildet haben. \\

\item Jetzt die Schokolade hinzufügen, schmelzen lassen und verrühren. Am Ende nochmal abschmecken. \\
\end{enumerate}
Tipp: Je öfter man nochmal Wasser hinzufügt und die Soße einkochen lässt, um so besser schmeckt sie, das darf gerne auch 2 Stunden dauern.\\

Am besten schmecken mir persönlich dazu al dente gekochte Linguine bzw. Bavette.
\end{minipage}
\vfill
\decothreeright \, \textbf{Arbeitszeit:} ca. 30 Min.	 / \textbf{Schwierigkeitsgrad:} normal	 / \decothreeleft \hfill Bewertung: \CIRCLE  \CIRCLE \CIRCLE \CIRCLE \LEFTcircle 
\ifoot{\href{http://www.chefkoch.de/rezepte/226491093168336/Gefuellte-Paprika-nach-Uroma-Susanne.html}{http://www.chefkoch.de/rezepte/Gefuellte-Paprika-nach-Uroma-Susanne.html}}
\section[Gefüllte Paprika nach Uroma Susanne]{\leafright\, Gefüllte Paprika nach Uroma Susanne \leafleft}
\begin{minipage}[t]{0.34\textwidth}
\vspace{0pt}\fbox{\includegraphics[width=1\linewidth]{./Bilder/gefuellte-paprika.png}}
\vspace{0.5cm}

\begin{small}
\begin{tabular}{R{1.7 cm} L{3.7cm} }
\multicolumn{2}{c}{\textbf{Zutaten für 4 Portionen}}\\ \toprule
10 &	 Paprikaschote(n) (Spitzpaprika), gelbe, je nach Größe evtl. mehr\\ \midrule[0.1mm]
500 g&	 Hackfleisch, gemischt\\ \midrule[0.1mm]
1 &	 Zwiebel(n)\\ \midrule[0.1mm]
1 Zehe/n&	 Knoblauch\\ \midrule[0.1mm]
 	& Salz und Pfeffer, schwarzer aus der Mühle\\ \midrule[0.1mm]
 	& Paprikapulver\\ \midrule[0.1mm]
100 g&	 Reis, gekochter\\ \midrule[0.1mm]
1 &	 Ei(er)\\ \midrule[0.1mm]
1 kl.& Dose/n	 Tomatenmark\\ \midrule[0.1mm]
500 ml&	 Gemüsebrühe\\ \midrule[0.1mm]
 etwas&	 Zucker\\ \midrule[0.1mm]
1 EL&	 Butter\\ \midrule[0.1mm]
1 EL&	 Mehl\\ \bottomrule
\end{tabular}
\end{small}
\end{minipage}
\hfill
\begin{minipage}[t]{0.58\textwidth}
\vspace{0pt}
\subsection*{Zubereitung}
\begin{enumerate}[leftmargin=*, itemindent=14pt]
\item Aus Hackfleisch, Reis, Ei, Zwiebel und Knoblauch einen Hackfleischteig herstellen und mit den Gewürzen abschmecken. In die Paprikaschoten füllen und aus dem Rest Hackfleischbällchen formen.
\item Die Butter in einem Topf schmelzen, das Mehl dazugeben und etwas anrösten. Mit Gemüsebrühe ablöschen, das Tomatenmark dazugeben und aufkochen lassen. Mit Salz, Pfeffer und etwas Zucker abschmecken. 
\item Die gefüllten Paprikaschoten und die Bällchen in die Soße geben und entweder auf dem Herd oder im Backofen 30-40 min schmoren lassen.
\item Dazu gibt es Reis. 
\end{enumerate}
Alternativ kann man auch Tomatenpüree statt des Tomatenmarks verwenden. Dann etwas weniger Brühe verwenden.
\end{minipage}
\vfill
\decothreeright \, \textbf{Arbeitszeit:} ca. 40 Min.	 / \textbf{Schwierigkeitsgrad:} normal	 / \decothreeleft \hfill Bewertung: \CIRCLE \CIRCLE \CIRCLE \CIRCLE \LEFTcircle
\ifoot{\href{http://www.chefkoch.de/rezepte/1754281285077184/Roemische-Zucchini.html}{http://www.chefkoch.de/rezepte/1754281285077184/Roemische-Zucchini.html}}
\section[Römische Zucchini]{\leafright\, Römische Zucchini \,\leafleft}
\begin{minipage}[t]{0.34\textwidth}
\vspace{0pt}
\fbox{\includegraphics[width=1\linewidth]{./Bilder/roemische-zucchini.png}}
\vspace{0.5cm}

\begin{small}
\begin{tabular}{R{1.6cm} L{3.8cm} }
\multicolumn{2}{c}{\textbf{Zutaten für 4 Portionen }}\\ \toprule
50 g&	 Speck, gewürfelt\\ \midrule[0.1mm]
10 ml&	 Olivenöl\\ \midrule[0.1mm]
1 &	 Zwiebel, gehackt\\ \midrule[0.1mm]
1 &	 Knoblauchzehe, gehackt\\ \midrule[0.1mm]
500 g&	 Tomaten, sehr saftig\\ \midrule[0.1mm]
1 EL&	 Oregano, gehackt\\ \midrule[0.1mm]
2 EL&	 Petersilie, gehackt\\ \midrule[0.1mm]
1 EL&	 Schnittlauch, gehackt\\ \midrule[0.1mm]
 	& Salz\\ \midrule[0.1mm]
60 g&	 Sahne\\ \midrule[0.1mm]
500 g&	 Zucchini, kleine\\ \midrule[0.1mm]
2 EL&	 Semmelbrösel\\ \midrule[0.1mm]
40 g&	 Parmesan\\ \midrule[0.1mm]
60 g&	 Sahne\\ \bottomrule
\end{tabular}
\end{small}
\end{minipage}
\hfill
\begin{minipage}[t]{0.58\textwidth}
\vspace{0pt}
\subsection*{Zubereitung}
\begin{enumerate}[leftmargin=*, itemindent=14pt]
\item Speck anbraten. Zwiebeln und Knoblauch zum Speck zugeben. Tomaten enthäuten, grob zerkleinern. Mit Oregano, Salz und Pfeffer würzen. Sahne hinzugeben. Köcheln lassen, bis eine dickliche Soße entsteht.
\item Von den Zucchinis die Enden abschneiden, längs halbieren und mit Salz bestreuen.

\item Semmelbrösel, Petersilie, Schnittlauch, (ggf. auch Dill), Parmesan und Sahne zu einer dicklichen Masse verrühren.

\item Tomatensoße in eine Auflaufform geben, die Zucchini darauf setzen. Auf die Zucchinihälften die Masse streichen. Bei 225-250°C ca. 20-25 Minuten im Backofen überbacken.
\end{enumerate}
Dazu passt sehr gut frisches Baguette.\\
Schmeckt auch super ohne Sahne.
\end{minipage}
\vfill
\decothreeright \, \textbf{Arbeitszeit:} 30 Min. / \textbf{Schwierigkeitsgrad:} normal \decothreeleft \hfill Bewertung:  \CIRCLE \CIRCLE \CIRCLE \CIRCLE  \LEFTcircle
\ifoot{\href{http://www.chefkoch.de/rezepte/1613999850871/Erbseneintopf.html}{http://www.chefkoch.de/rezepte/1613999850871/Erbseneintopf.html}}
\section[Erbseneintopf]{\leafright\, Erbseneintopf \leafleft}
\begin{minipage}[t]{0.34\textwidth}
\vspace{0pt}\fbox{\includegraphics[width=1\linewidth]{./Bilder/erbseneintopf.png}}
\vspace{0.5cm}

\begin{small}
\begin{tabular}{R{1.6 cm} L{3.8cm} }
\multicolumn{2}{c}{\textbf{Zutaten für 4 Portionen}}\\ \toprule
250 g&	 Erbsen, getrocknete\\ \midrule[0.1mm]
80 g&	 Speck, durchwachsen, gut geräuchert\\ \midrule[0.1mm]
1&	 Zwiebel, m.-groß\\ \midrule[0.1mm]
150 g&	 Karotten\\ \midrule[0.1mm]
100 g&	 Knollensellerie\\ \midrule[0.1mm]
300 g&	 Kartoffel(n), mehlig kochend\\ \midrule[0.1mm]
1 TL&	 Majoran, gerebelt, ca.\\ \midrule[0.1mm]
 	& Salz und Pfeffer, weiß aus der Mühle\\ \midrule[0.1mm]
4 	& Würstchen, Wiener\\ \midrule[0.1mm]
1 EL&	 Petersilie, gehackt\\ \bottomrule

\end{tabular}
\end{small}
\end{minipage}
\hfill
\begin{minipage}[t]{0.58\textwidth}
\vspace{0pt}
\subsection*{Zubereitung}
\begin{enumerate}[leftmargin=*, itemindent=14pt]

\item Die Erbsen waschen und über Nacht in ca. 1,5 L Wasser einweichen. 

\item Zwiebeln klein schneiden. Karotten, Sellerie und Kartoffeln schälen/putzen und in Würfelchen schneiden. Speck würfeln.

\item Die Zwiebel zusammen mit dem Speck in einem Topf auslassen. Karotten- und Selleriewürfel dazu geben und kurz anbraten lassen. Erbsen mit dem Einweichwasser dazugeben und alles ca. 1 Stunde köcheln lassen, dabei ab und an umrühren, damit nichts ansetzt. 

\item Dann die Kartoffeln und den Majoran zugeben, weiterkochen. Sind die Kartoffeln gar, den Eintopf würzen und die Würstchen darin heiß werden lassen. 

\item Wer möchte, kann den Eintopf auch, natürlich ohne die Würstchen, pürieren. Zum Schluss die Petersilie darüber streuen.

\end{enumerate}

Dazu schmeckt eine dicke Scheibe frisches kräftiges Bauernbrot.
\end{minipage}
\vfill
\decothreeright \, \textbf{Arbeitszeit:} ca. 20 Min.	 / \textbf{Schwierigkeitsgrad:} normal	 / \decothreeleft \hfill Bewertung: \CIRCLE \CIRCLE \CIRCLE \CIRCLE \Circle
\ifoot{\href{http://www.chefkoch.de/rezepte/1651531272966946/Schaschlik-wie-im-Kaukasus-grillen.html}{http://www.chefkoch.de/rezepte/Schaschlik-wie-im-Kaukasus-grillen.html}}
%\chapter{...alles...}
\section[Schaschlik wie im Kaukasus]{\leafright\, Schaschlik wie im Kaukasus \leafleft}
\begin{minipage}[t]{0.34\textwidth}
\vspace{0pt}
\fbox{\includegraphics[width=1\linewidth]{./Bilder/schaschlik-wie-im-kaukasus.png}}
\vspace{0.5cm}

\begin{small}
\begin{tabular}{R{1.6cm} L{3.8cm} }
\multicolumn{2}{c}{\textbf{Zutaten für 4 Portionen }}\\ \toprule
1 kg&	 Fleisch (Nackenfleisch von Schwein, Rind oder Lamm), frisch\\ \midrule[0.1mm]
3 &	 Gemüsezwiebel(n)\\ \midrule[0.1mm]
500 ml&	 Milch\\ \midrule[0.1mm]
1 Schuss&	 Essig\\ \midrule[0.1mm]
 	 &Salz und Pfeffer aus der Mühle\\ \midrule[0.1mm]
etwas&	 Tomatenmark\\ \midrule[0.1mm]
1 	& Kiwi\\ \bottomrule

\end{tabular}
\end{small}

\end{minipage}
\hfill
\begin{minipage}[t]{0.58\textwidth}
\vspace{0pt}
\subsection*{Zubereitung}
\begin{enumerate}[leftmargin=*, itemindent=14pt]
\item Den Schweinenacken in nicht zu kleine Würfel schneiden. Die Zwiebeln halbieren und in halbe Ringe schneiden. 

\item Tomatenmark, Salz, Pfeffer und einen Spritzer Essig mischen und mit der Hand schön in das Fleisch einmassieren. Alles in eine große Schüssel geben und mit Milch auffüllen. Zugedeckt über Nacht an einem kühlen Ort einziehen lassen.

\item Am nächsten Tag die Marinade probieren und evtl. mit Salz und Pfeffer nachwürzen. Ca. 2 Stunden vorm Grillen die Kiwi schälen und in kleine Stücke schneiden. In die Marinade geben und ebenfalls in das Fleisch einmassieren. Bitte die Kiwi nicht zu lange (max. 2 Stunden) in der Marinade lassen, da sonst das Fleisch zu weich wird und vom Spieß fällt.

\item Dann das Fleisch auf Spieße ziehen und grillen. Kurz vor dem Verzehr mit Essigwasser beträufeln. 
\end{enumerate}
Sehr lecker schmecken auch frische Zwiebelringe in leichtes Essigwasser eingelegt und zum Schaschlik serviert.
\end{minipage}
\vfill
\decothreeright \, \textbf{Arbeitszeit:} ca. 20 Min. / \textbf{Schwierigkeitsgrad:} normal \decothreeleft \hfill Bewertung: \CIRCLE \CIRCLE \CIRCLE \CIRCLE \LEFTcircle 
\ifoot{\href{http://www.chefkoch.de/rezepte/1621901269418083/Fladenbrotburger.html}{http://www.chefkoch.de/rezepte/1621901269418083/Fladenbrotburger.html}}
\section[Fladenbrotburger]{\leafright\, Fladenbrotburger \leafleft}
\begin{minipage}[t]{0.34\textwidth}
\vspace{0pt}\fbox{\includegraphics[width=1\linewidth]{./Bilder/fladenbrotburger.png}}
\vspace{0.5cm}

\begin{small}
\begin{tabular}{R{1.6cm} L{3.8cm} }
\multicolumn{2}{c}{\textbf{Zutaten für 4 Portionen}}\\ \toprule
1 &	 Fladenbrot\\ \midrule[0.1mm]
500 g&	 Hackfleisch, gemischt\\ \midrule[0.1mm]
3 große&	 Tomaten\\ \midrule[0.1mm]
1 Pck.&	 Schafskäse\\ \midrule[0.1mm]
100 g&	 Crème fraîche\\ \midrule[0.1mm]
 	& Salz und Pfeffer\\ \midrule[0.1mm]
 	& Paprikapulver\\ \midrule[0.1mm]
 	& Thymian, getrocknet\\ \bottomrule
\end{tabular}
\end{small}

\end{minipage}
\hfill
\begin{minipage}[t]{0.58\textwidth}
\vspace{0pt}
\subsection*{Zubereitung}
\begin{enumerate}[leftmargin=*, itemindent=14pt]
\item Hackfleisch in großer Pfanne braten und zerteilen, bis eine schöne Bräune erreicht ist. Mit Salz, Pfeffer, Paprikapulver und Thymian würzen.\\

\item Fladenbrot längs aufschneiden, das Hackfleisch gleichmäßig auf der Unterseite verteilen, einen kleinen Rand von 0,5 cm lassen. Nun mit Tomatenscheiben und dem aufgeschnittenen Schafskäse belegen, Crème fraîche in Klecksen darauf geben und die obere Brothälfte auflegen. Auf ein Backblech legen und bei 160 Grad Grad Umluft 12-15 min aufbacken.\\
\end{enumerate}
Lecker auch mit Mozzarella!
\end{minipage}
\vfill
\decothreeright \, \textbf{Arbeitszeit:} ca. 15 Min.	 / \textbf{Schwierigkeitsgrad:} simpel	 / \decothreeleft \hfill Bewertung: \Circle \Circle \Circle  \Circle \Circle
\ifoot{\href{http://www.chefkoch.de/rezepte/228531093692643/Das-schnellste-Huhn-der-Welt.html}{http://www.chefkoch.de/rezepte/228531093692643/Das-schnellste-Huhn-der-Welt.html}}
\section[Das schnellste Huhn der Welt]{\leafright\, Das schnellste Huhn der Welt \leafleft}
\begin{minipage}[t]{0.34\textwidth}
\vspace{0pt}\fbox{\includegraphics[width=1\linewidth]{./Bilder/das-schnellste-huhn-der-welt.png}}
\vspace{0.5cm}

\begin{small}
\begin{tabular}{R{1.6 cm} L{3.8cm} }
\multicolumn{2}{c}{\textbf{Zutaten für 2 Portionen}}\\ \toprule
2 &	 Hähnchenkeulen\\ \midrule[0.1mm]
 	& Salz und Pfeffer\\ \midrule[0.1mm]
2 Zehen&	 Knoblauch\\ \midrule[0.1mm]
2 EL&	 Öl\\ \midrule[0.1mm]
1/8 Liter&	 Wermut (Martini Bianco)\\ \midrule[0.1mm]
1 kl. Dose&	 Tomaten (Pizzatomaten)\\ \midrule[0.1mm]
100 g&	 Schlagsahne\\ \midrule[0.1mm]
 	& Thymian, frischen\\ \midrule[0.1mm]
 	& Paprikapulver\\ \midrule[0.1mm]
 	& Cayennepfeffer\\ \bottomrule
\end{tabular}
\end{small}
\end{minipage}
\hfill
\begin{minipage}[t]{0.58\textwidth}
\vspace{0pt}
\subsection*{Zubereitung}
\begin{enumerate}[leftmargin=*, itemindent=14pt]
\item Hähnchenkeulen waschen, trocken tupfen, halbieren, würzen mit Salz, Pfeffer, Paprika, Cayenne. Die Knoblauchzehen schälen, fein würfeln.\\

\item Öl in einer Pfanne erhitzen und die Hähnchenkeulen und den Knoblauch darin anbraten, mit Martini Bianco ablöschen.\\

\item Pizzatomaten zugeben und alles zugedeckt 20 min. köcheln.
Sahne zugießen, aufkochen und frischen Thymian zugeben.
Dieses Gericht isst man mit Nudeln oder Fladenbrot und einem frischen Blattsalat.
\end{enumerate}
\end{minipage}
\vfill
\decothreeright \, \textbf{Arbeitszeit:} ca. 30 Min.	 / \textbf{Schwierigkeitsgrad:} simpel	 / \decothreeleft \hfill Bewertung: \Circle  \Circle  \Circle  \Circle \Circle
\ifoot{\href{http://www.chefkoch.de/rezepte/293201108478436/Afrikanische-Erdnusssauce.html}{http://www.chefkoch.de/rezepte/293201108478436/Afrikanische-Erdnusssauce.html}}
\section[Afrikanische Erdnusssauce]{\leafright\, Afrikanische Erdnusssauce \leafleft}
\begin{minipage}[t]{0.34\textwidth}
\vspace{0pt}\fbox{\includegraphics[width=1\linewidth]{./Bilder/rezepte_afrikanische-erdnusssauce.png}}
\vspace{0.5cm}

\begin{small}
\begin{tabular}{R{1.6 cm} L{3.8cm} }
\multicolumn{2}{c}{\textbf{Zutaten für 2 Portionen}}\\ \toprule
2 EL&	 Öl\\ \midrule[0.1mm]
1& 	 Zwiebel, fein gehackt\\ \midrule[0.1mm]
1 kl. Dose&	 Tomatenmark\\ \midrule[0.1mm]
3/4 Liter&	 Wasser\\ \midrule[0.1mm]
1 bis 4 TL&	 Sambal Oelek\\ \midrule[0.1mm]
200 g&	 Erdnusscreme (Pindakaas)\\ \midrule[0.1mm]
 	& Salz\\ \midrule[0.5mm]
 & Couscous oder Reis\\ \midrule[0.1mm]
 & Hähnchenbrust oder Rindfleisch\\ \bottomrule
\end{tabular}
\end{small}
\end{minipage}
\hfill
\begin{minipage}[t]{0.58\textwidth}
\vspace{0pt}
\subsection*{Zubereitung}
\begin{enumerate}[leftmargin=*, itemindent=14pt]
\item Öl in einem Topf erhitzen, gehackte Zwiebel dazugeben, glasig braten lassen. Mit Wasser ablöschen. Salz, Sambal Olek und Tomatenmark unterrühren. Kurz aufkochen lassen. Von der Feuerstelle nehmen. Erdnusscreme einrühren und auf die Feuerstelle geben. Unter Rühren aufkochen lassen.\\

\leafNE\, oder
\item \begin{enumerate}[leftmargin=*, itemindent=18pt]

\item Das Fleisch (traditionell Rindfleisch) klein schneiden und zusammen mit den Zwiebeln in einem hohen Topf anbraten. Danach die oben angegebenen Zutaten dazugeben und auf nicht zu starker Flamme kochen lassen. Gelegentlich umrühren und mindestens 40 Minuten köcheln lassen (damit das Fleisch richtig durchgezogen ist mit dem Erdnussgeschmack). Vorsicht mit dem Wasser! Nur soviel dazu geben, dass die Sauce sämig aber nicht zu flüssig wird. 

\item Diese Sauce wird normalerweise sehr scharf gegessen, also seid grosszügig mit dem Cayenne-Pfeffer.

Noch ein Tipp: Es ist manchmal schwierig Pindakaas (Erknusscreme) in Deutschland zu bekommen - alternativ kann man Erdnussbutter nehmen (die findet ihr im "Nutella-Regal")\\
Um dem ganzen etwas mehr Rafinesse zu geben, serviert es doch mal mit Kokosnussreis - Wasser und Kokosnussmilch halb und halb mischen und den Reis dazugeben. 

\end{enumerate}
\end{enumerate}
\end{minipage}
\vfill
\decothreeright \, \textbf{Arbeitszeit:} ca. 10 Min.	 / \textbf{Schwierigkeitsgrad:} simpel	 / \decothreeleft \hfill Bewertung: \CIRCLE \CIRCLE \LEFTcircle  \Circle \Circle
\ifoot{\href{http://www.chefkoch.de/rezepte/1428661247733534/Toast-Hawaii-fruchtig-frisch.html}{http://www.chefkoch.de/rezepte/1428661247733534/Toast-Hawaii-fruchtig-frisch.html}}
\section[Toast Hawaii, fruchtig - frisch]{\leafright\, Toast Hawaii, fruchtig - frisch \leafleft}
\begin{minipage}[t]{0.34\textwidth}
\vspace{0pt}\fbox{\includegraphics[width=1\linewidth]{./Bilder/toast-hawaii.png}}
\vspace{0.5cm}

\begin{small}
\begin{tabular}{R{1.7 cm} L{3.7cm} }
\multicolumn{2}{c}{\textbf{Zutaten für 4 Portionen}}\\ \toprule
5 EL&	 Mayonnaise\\ \midrule[0.1mm]
100 g&	 Frischkäse\\ \midrule[0.1mm]
1 TL&	 Currypulver\\ \midrule[0.1mm]
 	& Salz\\ \midrule[0.1mm]
8 Scheiben&	 Toastbrot\\ \midrule[0.1mm]
8 Scheiben&	 Kochschinken\\ \midrule[0.1mm]
1 &	 Ananas, frisch\\ \midrule[0.1mm]
8 Scheiben	& mittelalter Gouda \\ \midrule[0.1mm]
2 EL&	 Zucker\\ \midrule[0.1mm]
20 g&	 Butter\\ \bottomrule
\end{tabular}
\end{small}
\end{minipage}
\hfill
\begin{minipage}[t]{0.58\textwidth}
\vspace{0pt}
\subsection*{Zubereitung}
\begin{enumerate}[leftmargin=*, itemindent=14pt]

\item Den Backofen auf 200 Grad (Umluft 180 Grad) vorheizen. 

\item Die Mayonnaise mit Frischkäse und Currypulver verrühren, salzen und pfeffern. 

\item Die Ananas schälen, längs vierteln und die holzige Mitte herausschneiden. Das Fruchtfleisch in 0,5 cm dicke Scheiben schneiden. In einer Pfanne etwa 20 g Butter zerlassen. Die Ananas in die Pfanne geben, mit dem Zucker bestreuen und leicht karamellisieren lassen.

\item Die Toastscheiben rösten, abkühlen lassen und mit der Mayonnaise-Frischkäse Mischung bestreichen. Erst je eine Scheibe Schinken, dann Ananasscheiben darauf legen. Mit den Käsescheiben abdecken und im Ofen ca. 10 Minuten backen, bis der Käse geschmolzen und leicht gebräunt ist. 


\end{enumerate}
Nach Belieben mit Currypulver bestäubt servieren.
\end{minipage}
\vfill
\decothreeright \, \textbf{Arbeitszeit:} ca. 20 Min.	 / \textbf{Schwierigkeitsgrad:} simpel	 / \decothreeleft \hfill Bewertung: \Circle  \Circle \Circle  \Circle \Circle
\input{./Rezepte/Hähnchenbrustfilet_gewickelt_Bohnen.tex}
\ifoot{\href{http://www.chefkoch.de/rezepte/1697101278066259/Gefuellte-Zucchini.html}{http://www.chefkoch.de/rezepte/1697101278066259/Gefuellte-Zucchini.html}}
\section[Gefüllte Zucchini]{\leafright\, Gefüllte Zucchini \,\leafleft}
mit Thunfisch und körnigem Frischkäse\\
\begin{minipage}[t]{0.34\textwidth}
\vspace{0pt}
\fbox{\includegraphics[width=1\linewidth]{./Bilder/gefuellte-zucchini.png}}
\vspace{0.5cm}

\begin{small}
\begin{tabular}{R{1.6cm} L{3.8cm} }
\multicolumn{2}{c}{\textbf{Zutaten für 2 Portionen }}\\ \toprule
3 &	 Zucchini, m. groß\\ \midrule[0.1mm]
1 &	 Zwiebel, m. groß\\ \midrule[0.1mm]
1 Dose&	 Thunfisch\\ \midrule[0.1mm]
200 g&	 Frischkäse, körniger\\ \midrule[0.1mm]
 n. B.&	 Kräuter\\ \midrule[0.1mm]
 n. B.&	 Parmesan\\ \midrule[0.1mm]
 	 &Salz und Pfeffer\\ \midrule[0.1mm]
 	& Öl\\ \bottomrule
\end{tabular}
\end{small}
\end{minipage}
\hfill
\begin{minipage}[t]{0.58\textwidth}
\vspace{0pt}
\subsection*{Zubereitung}
\begin{enumerate}[leftmargin=*, itemindent=14pt]
\item Die Zucchini halbieren und mit einem kleinen Löffel so aushöhlen, dass eine stabile Wand stehen bleibt. Die Zucchinihälften auf ein mit Backpapier belegtes Blech setzen und leicht salzen. 

\item Die Zwiebel fein hacken und in etwas Öl glasig anbraten. Die Hälfte des Zucchini-Inneren dazugeben und kurz mit anbraten. Mit Salz und Pfeffer würzen. 

\item In einer Schüssel den körnigen Frischkäse mit dem zerpflückten Thunfisch vermengen. Die Zwiebel-Zucchini-Mischung dazugeben und mit Kräutern nach Belieben würzen (Knoblauch, Schnittlauch, Petersilie - es passt fast alles). 

\item Mit einem Teelöffel die Masse in die Zucchinihälften füllen und bei 180° Umluft ca. 15 Minuten im vorgeheizten Ofen backen. Danach das Blech herausnehmen und etwas Parmesan über die Zucchinihälften streuen. Weitere 8-12 Minuten überbacken.
\end{enumerate}
\end{minipage}
\vfill
\decothreeright \, \textbf{Arbeitszeit:} 15 Min. / \textbf{Schwierigkeitsgrad:} normal \decothreeleft \hfill Bewertung:  \Circle  \Circle \Circle \Circle  \Circle
\ifoot{\href{http://www.chefkoch.de/rezepte/2542281398082259/Kabeljaufilet-auf-gruenem-Spargel-an-Orangen-Curry-Sauce.html}{http://www.chefkoch.de/rezepte/.../Kabeljaufilet-...-an-Orangen-Curry-Sauce.html}}
\section[Kabeljaufilet auf grünem Spargel an Orangen-Curry-Sauce]{\leafright\, Kabeljaufilet auf grünem Spargel an Orangen-Curry-Sauce \leafleft}
\begin{minipage}[t]{0.34\textwidth}
\vspace{0pt}\fbox{\includegraphics[width=1\linewidth]{./Bilder/kabeljaufilet-auf-gruenem-spargel.png}}
\vspace{0.5cm}

\begin{small}
\begin{tabular}{R{1.6 cm} L{3.8cm} }
\multicolumn{2}{c}{\textbf{Zutaten für 2 Portionen}}\\ \toprule

2 &	Kabeljaufilet(s) à ca. 150 g\\ \midrule[0.1mm]
500 g&	Spargel, grün\\ \midrule[0.1mm]
1 &	Orange(n)\\ \midrule[0.1mm]
30 g&	Butter\\ \midrule[0.1mm]
2 EL&	Olivenöl, nativ\\ \midrule[0.1mm]
1/2 EL&	Zucker\\ \midrule[0.1mm]
1/2 EL&	Curry\\ \midrule[0.1mm]
75 g&	Crème fraîche, oder saure Sahne\\ \midrule[0.1mm]
 	&Schnittlauch, in Röllchen\\ \midrule[0.1mm]
 	&Pfeffer, weiß aus der Mühle\\ \midrule[0.1mm]
 	&Salz\\ \midrule[0.1mm]
 	&Zitronensaft\\ \bottomrule
\end{tabular}
\end{small}
\end{minipage}
\hfill
\begin{minipage}[t]{0.58\textwidth}
\vspace{0pt}
\subsection*{Zubereitung}
\begin{enumerate}[leftmargin=*, itemindent=14pt]

\item Filets mit etwas Zitronensaft beträufeln, salzen und pfeffern. Orange schälen und filetieren, Saft auffangen. Das untere Drittel der Spargelstangen schälen, holziges Ende abschneiden und in ca. 3-4 cm lange, schräge Stücke schneiden. Dicke Spitzen längs halbieren. Spargelstücke in kochendem, gesalzenem Wasser mit etwas Butter und einer Prise Zucker ca. 8-10 Min. kochen. 

\item In der Zwischenzeit die 20 g Butter schmelzen, Zucker hinzufügen und leicht karamellisieren lassen. Currypulver einstreuen und mit dem Orangensaft ablöschen. Hitze reduzieren und Créme fraîche oder saure Sahne einrühren. Daneben in einer Pfanne das Olivenöl und die restliche Butter erhitzen und das gewürzte Kabeljau-Filet beidseitig ca. 3 Min. anbraten.

\item Auf vorgewärmten Tellern abgetropften Spargel mit den Orangenfilets anrichten und mit der Curry-Orangen-Sauce nappieren, mit Schnittlauch-Röllchen bestreuen und mit dem Fisch-Filet servieren. 

\end{enumerate}
Dazu passt ein leichter Sauvignon Blanc o.ä. und Weißbrot (Baguette oder Fladenbrot).
\end{minipage}
\vfill
\decothreeright \, \textbf{Arbeitszeit:} ca. 30 Min.	 / \textbf{Schwierigkeitsgrad:} normal	 / \decothreeleft \hfill Bewertung:\Circle \Circle \Circle \Circle \Circle %\CIRCLE \CIRCLE \LEFTcircle  \Circle \Circle
\ifoot{\href{https://www.chefkoch.de/rezepte/2133281343053838/Rinderrouladen-klassisch.html}{https://www.chefkoch.de/rezepte/2133281343053838/Rinderrouladen-klassisch.html}}
\section[Rinderrouladen klassisch]{\leafright\, Rinderrouladen klassisch \leafleft}
\begin{minipage}[t]{0.34\textwidth}
\vspace{0pt}
\fbox{\includegraphics[width=1\linewidth]{./Bilder/rinderrouladen-klassisch.png}}
\vspace{0.5cm}

\begin{small}
\begin{tabular}{R{1.6cm} L{3.8cm} }
\multicolumn{2}{c}{\textbf{Zutaten für 2 Portionen }}\\ \toprule

8 & 	Roulade(n) vom Rind \\ \midrule[0.1mm]
5  &	Zwiebel(n) \\ \midrule[0.1mm]
4  &	Gewürzgurke(n) \\ \midrule[0.1mm]
12 &Scheibe/n 	Frühstücksspeck \\ \midrule[0.1mm]
4 EL& 	Senf, mittelscharfer \\ \midrule[0.1mm]
1 Stück(e) &	Knollensellerie \\ \midrule[0.1mm]
1  	& Möhre(n) \\ \midrule[0.1mm]
1/2 Stange/n &	Lauch \\ \midrule[0.1mm]
1/2 Flasche &	Rotwein, guter \\ \midrule[0.1mm]
  	& Salz \\ \midrule[0.1mm]
  	& Pfeffer \\ \midrule[0.1mm]
1/2 Liter &	Rinderfond, kräftiger \\ \midrule[0.1mm]
 TL &	Speisestärke \\ \midrule[0.1mm]
1 Schuss &	Gurkenflüssigkeit \\ \midrule[0.1mm]
2 EL &	Butterschmalz  \\ \bottomrule

\end{tabular}
\end{small}
\end{minipage}
\hfill
\begin{minipage}[t]{0.58\textwidth}
\vspace{0pt}
\subsection*{Zubereitung}
\begin{enumerate}[leftmargin=*, itemindent=14pt]
\item Die Rinderrouladen aufrollen, waschen und mit Küchenkrepp trockentupfen. Zwiebeln in Halbmonde, Gurken in Längsstreifen schneiden, Schere und Küchengarn bereitstellen. 

\item Die ausgebreiteten Rouladen dünn mit Senf bestreichen, salzen und pfeffern, auf jede Roulade mittig in der Länge ca. 1/2 Zwiebel und 1 1/2 Scheiben Frühstücksspeck sowie 1/2 (evtl. mehr) Gurke verteilen. Nun von beiden Längsseiten etwas einschlagen, dann aufrollen und mit dem Küchengarn wie ein Postpaket verschnüren.

\item In einer Pfanne das Butterschmalz heiß werden lassen und die Rouladen dann rundherum darin anbraten, herausnehmen und in einen Schmortopf umfüllen.

\item Den Sellerie, die restliche Zwiebel, das Lauch und die Möhren kleinschneiden und in der Pfanne anbraten. Sobald sie halbwegs "blond" sind, kurz rühren, eine sehr dünne Schicht vom Rotwein angießen, nicht mehr rühren und die Flüssigkeit verdampfen lassen. Sobald das Gemüse dann wieder trockenbrät, wieder eine Schicht angießen, kurz rühren und weiter verdampfen lassen. Dies wiederholen, bis die 1/2 Flasche Wein aufgebraucht ist. Auf diese Art wird das Röstgemüse sehr braun (gut für den Geschmack und die Farbe der Soße) aber nicht trocken. Am Schluss mit der Fleischbrühe, etwas Salz und Pfeffer und einem guten Schuss Gurkensud auffüllen und dann in den Schmortopf zu den Rouladen geben. Den Topf entweder auf kleiner Flamme oder bei ca. 160 Grad im Backofen für 1 1/2 Stunden schmoren lassen. Ab und zu evtl. etwas Flüssigkeit zugießen.

\item Nach 1 1/2 Stunden testen, ob die Rouladen weich sind (einfach mal mit den Kochlöffel ein bisschen draufdrücken, sie sollten sich willig eindrücken lassen, wenn nicht, nochmal eine halbe Stunde weiterschmoren) und dann vorsichtig aus dem Topf heben, warm stellen.

\item Die Soße durch ein Sieb geben, aufkochen. Ca. 1 El Senf mit etwas Wasser und der Speisestärke gut verrühren und in die kochende Soße nach und nach unter Rühren eingießen, bis die gewünschte Konsistenz erreicht ist. Die Soße evtl. nochmal mit Salz, Pfeffer, Rotwein, Gurkensud abschmecken. 

\end{enumerate}
\end{minipage}
\vfill
\decothreeright \, \textbf{Arbeitszeit:} ca. 1 Std. / \textbf{Koch-Backzeit:} ca. 2 Std. /\textbf{Schwierigkeitsgrad:} pfiffig \decothreeleft \hfill \\ Bewertung:  \Circle  \Circle \Circle \Circle \Circle



\chapter{Rezepte ohne Fleisch}
\ifoot{\href{http://www.chefkoch.de/rezepte/1110251217076836/Suesskartoffelsticks-mit-Honig-Sesam-Dip.html}{http://www.chefkoch.de/rezepte/1110251217076836/Suesskartoffelsticks-mit-Honig-Sesam-Dip.html}}
\section[Süßkartoffelsticks mit Honig - Sesam - Dip]{\leafright\, Süßkartoffelsticks mit Honig - Sesam - Dip \leafleft}
\begin{minipage}[t]{0.34\textwidth}
\vspace{0pt}\fbox{\includegraphics[width=1\linewidth]{./Bilder/suesskartoffelsticks-mit-honig-sesam-dip.png}}
\vspace{0.5cm}

\begin{small}
\begin{tabular}{R{1.6 cm} L{3.8cm} }
\multicolumn{2}{c}{\textbf{Zutaten für 2 Portionen}}\\ \toprule
350 g&	Süßkartoffel(n), geschält\\ \midrule[0.1mm]
5 EL&	Honig\\ \midrule[0.1mm]
1 1/2 EL&	Sesam\\ \midrule[0.1mm]
2 EL&	Zitronensaft\\ \midrule[0.1mm]
 	&Fett zum Frittieren\\ \bottomrule
\end{tabular}
\end{small}
\end{minipage}
\hfill
\begin{minipage}[t]{0.58\textwidth}
\vspace{0pt}
\subsection*{Zubereitung}
\begin{enumerate}[leftmargin=*, itemindent=14pt]
\item Die Kartoffeln in pommesähnliche Sticks schneiden und trocken tupfen. 

\item Den Sesam in einer trockenen Pfanne rösten. Dann mit dem Honig und dem Zitronensaft verrühren. \\- Siehe \nameref{sec:4 Schritte zum ideal gerösteten Sesam} -

\item Die Kartoffelsticks im heißen Fett knusprig backen und anschließend entweder mit der Honigsoße überziehen oder die Soße als Dip dazu reichen.
\end{enumerate}
\end{minipage}
\vfill
\decothreeright \, \textbf{Arbeitszeit:} ca. 20 Min.	 / \textbf{Schwierigkeitsgrad:} simpel	 / \decothreeleft \hfill Bewertung: \Circle  \Circle  \Circle  \Circle \Circle
%\input{./Rezepte/Zucchini_Wedges_mit_Pesto_Dip.tex}
\ifoot{\href{http://www.chefkoch.de/rezepte/789361182263001/Brennnessel-Kartoffel-Suppe.html}{http://www.chefkoch.de/rezepte/789361182263001/Brennnessel-Kartoffel-Suppe.html}}
\section[Brennnessel - Kartoffel - Suppe]{\leafright\, Brennnessel - Kartoffel - Suppe \leafleft}
\begin{minipage}[t]{0.34\textwidth}
\vspace{0pt}
\fbox{\includegraphics[width=1\linewidth]{./Bilder/brennnessel-kartoffel-suppe.png}}
\vspace{0.5cm}

\begin{small}
\begin{tabular}{R{1.8cm} L{3.6cm} }
\multicolumn{2}{c}{\textbf{Zutaten für 4 Portionen }}\\ \toprule
3 Handvoll& Brennnesseln, frisch\\ \midrule[0.1mm]
500 g& Kartoffeln\\ \midrule[0.1mm]
200 g& Karotten, klein gewürfelt\\ \midrule[0.1mm]
1& große Zwiebel\\ \midrule[0.1mm]
1 Stück& Butter\\ \midrule[0.1mm]
1 El& Mehl (Vollkorn)\\ \midrule[0.1mm]
2 El& Gemüsebrühe\\ \midrule[0.1mm]
1\,\textonehalf\;l &Wasser\\ \midrule[0.1mm]
&Majoran\\ \midrule[0.1mm]
&Salz\\ \midrule[0.1mm]
&Pfeffer\\ \bottomrule
\end{tabular}
\end{small}

\end{minipage}
\hfill
\begin{minipage}[t]{0.58\textwidth}
\vspace{0pt}
\subsection*{Zubereitung}
\begin{enumerate}[leftmargin=*, itemindent=14pt]
\item Die Kartoffeln schälen und klein schneiden. Die Möhre säubern und in kleine Stücke schneiden. Beides zusammen mit der gekörnten Gemüsebrühe in das Wasser geben und weich kochen.\\

\item Die Zwiebeln schälen, in kleine Stücke schneiden und im Öl anbraten. Dann mit dem Mehl bestäuben und eine Zwiebel-Mehlschwitze bereiten. Langsam Wasser zugeben und alles schnell verrühren, damit es nicht klumpt, dann in den Topf zu den Kartoffeln geben. Jetzt teilweise die Kartoffeln und die Möhre mit einem Stampfer zerdrücken und weiterhin köcheln lassen, damit die Suppe sämig wird.\\

\item Die gewaschenen und fein gewiegten Brennnesselspitzen einrühren und ca. $1/4$ Stunde ziehen lassen. Mit Salz und Majoran abschmecken. Nach Belieben die Sahne dazu geben und servieren.
\end{enumerate}
\end{minipage}
\vfill
\decothreeright \, \textbf{Arbeitszeit:} ca. 15 Min. / \textbf{Schwierigkeitsgrad:} simpel \decothreeleft \hfill Bewertung: \CIRCLE  \CIRCLE \CIRCLE \CIRCLE \CIRCLE 
\ifoot{\href{https://www.cookingchanneltv.com/recipes/david-rocco/lentil-and-tomato-soup-1960593}{https://www.cookingchanneltv.com/recipes/david-rocco/lentil-and-tomato-soup-1960593}}
\section[Linsen- und Tomatensuppe]{\leafright\,Linsensuppe mit Tomaten \leafleft}
\begin{minipage}[t]{0.34\textwidth}
\vspace{0pt}
\fbox{\includegraphics[width=1\linewidth]{./Bilder/Linsen-und-Tomaten-Suppe.png}}
\vspace{0.5cm}

\begin{small}
\begin{tabular}{R{1.6cm} L{3.8cm} }
\multicolumn{2}{c}{\textbf{Zutaten für 4 Portionen }}\\ \toprule
1/4 Tasse (50 ml) & Olivenöl extra Vergine \\ \midrule[0.1mm]
2 Zehen & Knoblauch\\ \midrule[0.1mm]
1 Bündel & frische Petersilie\\ \midrule[0.1mm]
1 Dose (441 ml) g& ganze Tomaten aus der Dose\\ \midrule[0.1mm]
2 & frische Chilischoten\\ \midrule[0.1mm]
1 Dose (540 ml) & Linsen\\ \midrule[0.1mm]
& Salz, Pfeffer\\ \midrule[0.1mm]
& Wasser (optional)\\ \midrule[0.1mm]
25 g& Gersten-Malz\\ \bottomrule
\end{tabular}
\end{small}
\end{minipage}
\hfill
\begin{minipage}[t]{0.58\textwidth}
\vspace{0pt}
\subsection*{Zubereitung}
\begin{enumerate}[leftmargin=*, itemindent=14pt]
\item Knoblauch, Chilischoten und Petersilie klein schneiden. Linsen abtropfen lassen, dabei  einen Teil der Flüssigkeit auffangen und Linsen abspülen.
\item Das Olivenöl in einer Soßenpfanne (oder einem Topf) erhitzen. Knoblauch, Chili und Petersilie für wenige Minuten anbraten. 
\item Die Tomaten hinzugeben und mit einem Holzlöffel die Tomaten in kleine Stücke zerdrücken. Die Linsen, Salz und Pfeffer, sowie die Flüssigkeit der Linsen und ggf. Wasser hinzugeben.
\item Alles für 15 bis 20 Minuten köcheln lassen, bis die Suppe eingedickt ist. 

\end{enumerate}
\end{minipage}
\vfill
\decothreeright \, \textbf{Arbeitszeit:} ca. 30 Min. / \textbf{Schwierigkeitsgrad:} simpel \decothreeleft \hfill Bewertung: \CIRCLE  \CIRCLE \CIRCLE  \CIRCLE \CIRCLE
\ifoot{\href{http://www.chefkoch.de/rezepte/237711096640386/Blumenkohl-gebraten.html}{chefkoch.de/Blumenkohl-gebraten} \& \href{http://www.chefkoch.de/rezepte/1686711276938595/Blumenkohl-im-Bierteig-ausgebacken.html}{chefkoch.de/Blumenkohl-im-Bierteig-ausgebacken}}
\section[Blumenkohl gebraten mit Bechamelsoße]{\leafright\, Blumenkohl gebraten mit Bechamelsoße \,\leafleft}
\begin{minipage}[t]{0.34\textwidth}
\vspace{0pt}
\fbox{\includegraphics[width=1\linewidth]{./Bilder/blumenkohl_gebraten.png}}
\vspace{0.5cm}

\begin{small}
\begin{tabular}{R{1.2cm} L{4.2cm} }
\multicolumn{2}{c}{\textbf{Zutaten für 4 Portionen }}\\ \toprule
1 &	 Blumenkohl\\ \midrule[0.1mm]
3 &	 Ei(er)\\ \midrule[0.1mm]
 &	 Semmel(n), geriebene Paniermehl\\ \midrule[0.1mm]
 &	 Salz, Pfeffer, Muskat, Worcestershiresauce\\ \midrule[0.1mm]
 &	 Butter  Öl\\ \midrule[0.2mm]\addlinespace
\multicolumn{2}{c}{Alternative: Bierteig}\\  \midrule[0.1mm] 
125 ml& Bier, hell\\  \midrule[0.1mm]
2& Eier\\  \midrule[0.1mm]
140 g& Mehl\\ \midrule[0.1mm]
& Salz \\ \midrule[0.1mm]
1 Tl.& Öl \\ \bottomrule
\end{tabular}
\end{small}
\end{minipage}
\hfill
\begin{minipage}[t]{0.58\textwidth}
\vspace{0pt}
\subsection*{Zubereitung}
\begin{enumerate}[leftmargin=*, itemindent=14pt]
\item Den Blumenkohl wie gewohnt in kleine Röschen teilen und bissfest in Salzwasser kochen. Kochwasser aufheben. Soweit abkühlen lassen dass man diesen zum panieren in den Händen halten kann ohne sich zu verbrennen. 

\item Panieren mit Ei und Paniermehl (wie ein Schnitzel), vielleicht das Ei vorher noch mit Pfeffer würzen. Anschließend in der Pfanne Butter erhitzen und den Blumenkohl schön goldbraun ringsherum braten.

\item Ein großes Stück Margarine im Topf erhitzen und das Mehl darin anschwitzen. Anschließend unter Rühren das Blumenkohlwasser hinzufügen. Dann soviel Milch hinzufügen bis sich eine sämige, nicht zu feste Soße gebildet hat. In die Soße kommt noch etwas Zitronensaft, Muskat und Worchester-Soße. Mit Salz abschmecken und fertig. 
\end{enumerate}

\subsection*{Alternative: Bierteig}
\begin{enumerate}[leftmargin=*, itemindent=14pt]
\item Für den Bierteig die Eiweiße steif schlagen, dann in den Kühlschrank stellen.

\item Mehl, Bier, Öl, Salz, Muskatnuss und Eigelb schnell miteinander zu einem geschmeidigen Teig verrühren, nicht zu lange rühren, sonst wird der Teig später zäh. Den Teig etwas rasten lassen, dann zügig, aber vorsichtig! den Eischnee unterheben.

\item Die Blumenkohlröschen durch den Teig ziehen und im heißen Öl schwimmend ausbacken, bis sie schön goldgelb sind.
\end{enumerate}
\end{minipage}
\vfill
\decothreeright \, \textbf{Arbeitszeit:} 30 Min. / \textbf{Schwierigkeitsgrad:} simpel \decothreeleft \hfill Bewertung:  \CIRCLE \CIRCLE \CIRCLE \CIRCLE  \Circle
\ifoot{\href{http://www.chefkoch.de/rezepte/825261187174998/Erbsensuppe.html}{http://www.chefkoch.de/rezepte/825261187174998/Erbsensuppe.html}}
\section[Erbsensuppe]{\leafright\, Erbsensuppe \,\leafleft}
\begin{minipage}[t]{0.34\textwidth}
\vspace{0pt}
\fbox{\includegraphics[width=1\linewidth]{./Bilder/erbsensuppe.png}}
\vspace{0.5cm}

\begin{small}
\begin{tabular}{R{1.6cm} L{3.8cm} }
\multicolumn{2}{c}{\textbf{Zutaten für 4 Portionen }}\\ \toprule
1 Stange&	 Lauch\\ \midrule[0.1mm]
250 g&	 Kartoffel(n) (am besten mehlig kochende)\\ \midrule[0.1mm]
50 g&	 Karotte(n)\\ \midrule[0.1mm]
2 El&	 Öl\\ \midrule[0.1mm]
2 Würfel&	 Fleischbrühe\\ \midrule[0.1mm]
600 g&	 Erbsen (TK)\\ \midrule[0.1mm]
1 Tl&	 Majoran\\ \midrule[0.1mm]
50 ml&	 Schlagsahne\\ \midrule[0.1mm]
 	& Salz\\ \midrule[0.1mm]
 	& Pfeffer\\ \midrule[0.1mm]
 evtl.&	 Würstchen nach Belieben\\ \midrule[0.1mm]
 	& Petersilie zum Garnieren\\ \midrule[0.1mm]
1 Liter&	 Wasser\\ \bottomrule
\end{tabular}
\end{small}
\end{minipage}
\hfill
\begin{minipage}[t]{0.58\textwidth}
\vspace{0pt}
\subsection*{Zubereitung}
\begin{enumerate}[leftmargin=*, itemindent=14pt]
\item Lauch waschen, putzen und in kleine Stücke schneiden. Kartoffeln und Karotten schälen und würfeln. 

\item Lauchstücke und Kartoffel- und Karottenstücke in einem großen Topf in heißem Öl andünsten. 1 Liter Wasser aufgießen und aufkochen lassen. Die 2 Würfel Fleischsuppe und die gefrorenen Erbsen und den Majoran hinzugeben und bei schwacher Hitze ca. 15 Minuten köcheln lassen. Suppe pürieren. Die Sahne zugeben und mit Salz und Pfeffer abschmecken. Eventuell Würstchen nach Belieben in Scheiben schneiden oder ganz zufügen und noch 5 Minuten erhitzen lassen, aber nicht mehr kochen. Mit Petersilie garniert servieren.
\end{enumerate}
\end{minipage}
\vfill
\decothreeright \, \textbf{Arbeitszeit:} 30 Min. / \textbf{Schwierigkeitsgrad:} simpel \decothreeleft \hfill Bewertung:   \CIRCLE \CIRCLE \CIRCLE \CIRCLE  \LEFTcircle
\ifoot{\href{http://www.chefkoch.de/rezept-anzeige.php?ID=819391186467176}{http://www.chefkoch.de/rezept-anzeige.php?ID=819391186467176}}
\section[Linsenbolognese mit Pasta und Frühlingszwiebeln]{\leafright\, Linsenbolognese mit Pasta und Frühlingszwiebeln \leafleft}
\begin{minipage}[t]{0.34\textwidth}
\vspace{0pt}
\fbox{\includegraphics[width=1\linewidth]{./Bilder/linsenbolognese-mit-pasta-und-fruehlingszwiebeln.jpg}}
\vspace{0.5cm}

\begin{small}
\begin{tabular}{R{1.8cm} L{3.6cm} }
\multicolumn{2}{c}{\textbf{Zutaten für 3 Portionen }}\\ \toprule
100 g&	Linsen, rote\\ \midrule[0.1mm]
1 &	Zwiebeln, gewürfelt\\ \midrule[0.1mm]
1 &Dose	Tomaten, stückige\\ \midrule[0.1mm]
300 ml&	Hühnerbrühe\\ \midrule[0.1mm]
3 &	Frühlingszwiebeln, in feine Ringe geschnitten\\ \midrule[0.1mm]
1 &	Karotte, in feine Ringe geschnitten\\ \midrule[0.1mm]
50 ml&	Sahne  oder\\
70 ml&Milch\\ \midrule[0.1mm]
1& Schuss	Balsamico\\ \midrule[0.1mm]
1 &	Knoblauchzehe(n), durchgepresst\\ \midrule[0.1mm]
 &	Cayennepfeffer\\ \midrule[0.1mm]
 &	Parmesan, frisch gerieben\\ \midrule[0.1mm]
250 g&	Nudeln\\ \midrule[0.1mm]
 &	Salz und Pfeffer\\ \bottomrule
\end{tabular}
\end{small}



\end{minipage}
\hfill
\begin{minipage}[t]{0.58\textwidth}
\vspace{0pt}
\subsection*{Zubereitung}
\begin{enumerate}[leftmargin=*, itemindent=14pt]

\item Nudeln "`al dente"' kochen. 

\item Zwiebeln in Öl anbraten, dann die Linsen dazu geben und kurz anschwitzen. Heiße Hühnerbrühe angießen und aufkochen lassen. Bei mittlerer Hitze 10 - 15 Minuten köcheln lassen, bis die Linsen gar sind. Die Tomaten zugeben und aufkochen lassen. Frühlingszwiebeln, Knoblauch und Sahne hinzu geben und mit Salz, Pfeffer, Balsamico und Cayennepfeffer abschmecken. 

\item Mit den Nudeln und mit Parmesan bestreut servieren.
\end{enumerate}
\end{minipage}
\vfill
\decothreeright \, \textbf{Arbeitszeit:} ca. 20 Min. / \textbf{Schwierigkeitsgrad:} normal \decothreeleft \hfill Bewertung: \CIRCLE  \CIRCLE \CIRCLE \CIRCLE \CIRCLE 
\ifoot{\href{https://www.chefkoch.de/rezepte/420751132752442/Schnelle-Gemuese-Burritos.html}{https://www.chefkoch.de/rezepte/420751132752442/Schnelle-Gemuese-Burritos.html}}

\section[Schnelle Gemüse - Burritos]{\leafright\, Schnelle Gemüse - Burritos \leafleft}
\begin{minipage}[t]{0.34\textwidth}
\vspace{0pt}
\fbox{\includegraphics[width=1\linewidth]{./Bilder/schnelle_gemuese_burritos.png}}
\vspace{0.5cm}

\begin{small}
\begin{tabular}{R{1.6cm} L{3.8cm} }
\multicolumn{2}{c}{\textbf{Zutaten für 4 Portionen }}\\ \toprule 

8 & 	Tortilla(s)\\ \midrule[0.1mm]
1  &	Zwiebel(n)\\ \midrule[0.1mm]
1 Zehe/n &	Knoblauch\\ \midrule[0.1mm]
2  &	Karotte(n)\\ \midrule[0.1mm]
1  &	Paprikaschote(n), rote\\ \midrule[0.1mm]
1  &	Paprikaschote(n), gelbe\\ \midrule[0.1mm]
1  &	Paprikaschote(n), grüne\\ \midrule[0.1mm]
1 Dose/n &	Mais\\ \midrule[0.1mm]
1 Dose/n &	Kidneybohnen\\ \midrule[0.1mm]
1 Becher &	Crème fraîche\\ \midrule[0.1mm]
  &	Salz und Pfeffer und Cayennepfeffer\\ \midrule[0.1mm]
  &	Oregano \\ \bottomrule

\end{tabular}
\end{small}

\end{minipage}
\hfill
\begin{minipage}[t]{0.58\textwidth}
\vspace{0pt}
\subsection*{Zubereitung}
\begin{enumerate}[leftmargin=*, itemindent=14pt]

    \item Die Tortillas nach Belieben leicht anbraten.

    \item Die Zwiebel und die Knoblauchzehe schälen und fein hacken, die Karotte in Scheiben schneiden und alles in etwas Öl in einer großen Pfanne andünsten.

    \item Währenddessen die Paprikaschoten entkernen, würfeln und dann etwas mitdünsten.

    \item Kidneybohnen und Mais abgießen und unter das Gemüse mengen.

    \item Auf jede Tortilla nun etwa 2-3 El Gemüse und 2-3 Tl Creme fraiche geben, die Seiten darüber zusammenschlagen und die Unterkante hochklappen. Am besten in Alufolie wickeln, damit man sie sauber essen kann.

    \item Zum warm halten am besten den Ofen kurz anheizen und die Burritos darin bei geschlossener Tür bis zum Verzehr lagern (aber bitte nicht länger als 20 min!)


    \leafNE\, Überbacken

    \item Alternativ können die Burritos auch in eine Auflaufform geschichtet werden und bei 200°C für 12 bis 15 min mit Käse überbacken werden. 

\end{enumerate}
Sehr lecker schmecken auch frische Zwiebelringe in leichtes Essigwasser eingelegt und zum Schaschlik serviert.
\end{minipage}
\vfill
\decothreeright \, \textbf{Arbeitszeit:} ca. 20 Min. / \textbf{Schwierigkeitsgrad:} normal \decothreeleft \hfill Bewertung: \CIRCLE \CIRCLE \CIRCLE \CIRCLE \LEFTcircle 
\ifoot{\href{http://www.chefkoch.de/rezepte/1122371218540418/Karotten-Curry-Sugo.html}{http://www.chefkoch.de/rezepte/1122371218540418/Karotten-Curry-Sugo.html}}
\section[Karotten - Curry - Sugo]{\leafright\, Karotten - Curry - Sugo \,\leafleft}
\begin{minipage}[t]{0.34\textwidth}
\vspace{0pt}
\fbox{\includegraphics[width=1\linewidth]{./Bilder/karotten-curry-sugo.png}}
\vspace{0.5cm}

\begin{small}
\begin{tabular}{R{1.6cm} L{3.8cm} }
\multicolumn{2}{c}{\textbf{Zutaten für 4 Portionen }}\\ \toprule
400 g&	 Karotten\\ \midrule[0.1mm]
1 kleine&	 Zucchini\\ \midrule[0.1mm]
1& 	 Zwiebel\\ \midrule[0.1mm]
1 &	 Chilischote\\ \midrule[0.1mm]
2 &	 Knoblauchzehen\\ \midrule[0.1mm]
20 g&	 Ingwer\\ \midrule[0.1mm]
 etwas&	 Weißwein\\ \midrule[0.1mm]
250 ml&	 Sauerrahm\\ \midrule[0.1mm]
200 g&	 Tomatenmark\\ \midrule[0.1mm]
 &	 Öl\\ \midrule[0.1mm]
 &	 Salz und Pfeffer\\ \midrule[0.1mm]
1 TL&	 Currypulver\\ \midrule[0.1mm]
 	& Parmesan, frisch gerieben\\ \midrule[0.1mm]
400 g&	 Nudeln \\ \bottomrule
\end{tabular}
\end{small}
\end{minipage}
\hfill
\begin{minipage}[t]{0.58\textwidth}
\vspace{0pt}
\subsection*{Zubereitung}
\begin{enumerate}[leftmargin=*, itemindent=14pt]
\item Die Nudeln bissfest kochen. Die Karotten hobeln, Zwiebel und Zucchini würfeln. Chili, Knoblauch und Ingwer fein hacken. 

\item Die Zwiebeln in Öl anschwitzen, Karotten und Zucchini einrühren und anrösten. Knoblauch, Ingwer und Chili einrühren und kurz mitrösten. Currypulver dazu geben, kurz weiter rösten, dann mit etwas Wein ablöschen. Sauerrahm und Tomatenmark dazu geben und halb zugedeckt köcheln lassen, bis die Karotten weich sind. 

\item Den Sugo mit Salz und Pfeffer abschmecken, mit den Nudeln mischen und sofort mit Parmesan servieren.
\end{enumerate}
\end{minipage}
\vfill
\decothreeright \, \textbf{Arbeitszeit:} 30 Min. / \textbf{Schwierigkeitsgrad:} simpel \decothreeleft \hfill Bewertung:  \CIRCLE \CIRCLE \CIRCLE  \Circle \Circle
\ifoot{\href{http://www.chefkoch.de/rezepte/341201118049854/Pikante-Reistorte.html}{http://www.chefkoch.de/rezepte/341201118049854/Pikante-Reistorte.html}}
\section[Pikante Reistorte]{\leafright\, Pikante Reistorte \leafleft}
\begin{minipage}[t]{0.33\textwidth}
\vspace{0pt}
\fbox{\includegraphics[width=1\linewidth]{./Bilder/pikante_reistorte.png}}
\vspace{0.5cm}

\begin{tabular}{r p{3.6cm} }
\multicolumn{2}{c}{\textbf{Zutaten für 4 Portionen }}\\ \toprule
400 g& Hähnchenbrustfilet (oder Putenschnitzel)\\ \midrule
2 &	 Paprikaschote(n)\\ \midrule
2 &	 Tomate(n)\\ \midrule
1 &	 Zwiebel(n)\\ \midrule
4 &	 Ei(er)\\ \midrule
1 Becher& Crème fraîche (150g)\\ \midrule
200 g&	 geriebener Käse (Emmentaler / Gouda)\\ \midrule
250 g&	 Reis\\ \midrule
 	& Salz und Pfeffer\\ \midrule
 	& Tabasco\\ \midrule
 	& Sonnenblumenöl\\ \bottomrule
\end{tabular}

\end{minipage}
\hfill
\begin{minipage}[t]{0.60\textwidth}
\vspace{0pt}
\subsection*{Zubereitung}
Hähnchenbrust in kleine Würfel schneiden, Paprika und Zwiebel würfeln. Die Hähnchenbrust in Sonnenblumenöl anbraten, Zwiebeln hinzugeben und glasig werden lassen. Die Paprikawürfel kurz mitbraten (sollen noch gut Biss haben). Alles mit Salz, Pfeffer, Tabasco gut würzen. Aus dem Topf nehmen.\\

In dem übriggebliebenen Öl den Reis glasig dünsten, Wasser hinzugeben (1 Teil Reis / 2 Teile Wasser), salzen und quellen lassen (sollte noch etwas Biss haben).\\

In der Zwischenzeit Eier mit Creme Fraiche und 150 g Käse vermischen. Dann Hähnchen, Paprika und Reis hinzugeben. Gut mischen und noch mal kräftig mit Salz, Pfeffer und Tabasco abschmecken. In eine Springform füllen und den restlichen Käse darüber streuen. Tomaten in Scheiben schneiden und auf der Reismasse verteilen. Bei 200 Grad 20 Minuten backen.\\

Schmeckt besonders gut, wenn man die Torte mit einer fruchtigen Tomatensoße serviert. Ist gut vorzubereiten und schmeckt auch aufgewärmt gut.
\end{minipage}
\vfill
\decothreeright \, \textbf{Arbeitszeit:} ca. 30 Min. / \textbf{Schwierigkeitsgrad:} normal \decothreeleft \hfill Bewertung: \CIRCLE \LEFTcircle  \Circle  \Circle \Circle
\ifoot{\href{http://www.chefkoch.de/rezepte/1200081225877998/Bunte-Frittata.html}{http://www.chefkoch.de/rezepte/1200081225877998/Bunte-Frittata.html}}
\section[Bunte Frittata]{\leafright\, Bunte Frittata \leafleft}
\begin{minipage}[t]{0.34\textwidth}
\vspace{0pt}\fbox{\includegraphics[width=1\linewidth]{./Bilder/bunte_frittata.png}}
\vspace{0.5cm}

\begin{small}
\begin{tabular}{R{1.6cm} L{3.8cm} }
\multicolumn{2}{c}{\textbf{Eine Quicheform 24 cm }}\\ \toprule
500 g&	 Karotten\\ \midrule[0.1mm]
250 g&	 Brokkoli\\ \midrule[0.1mm]
4 &	 Eier\\ \midrule[0.1mm]
75 g&	 Sauerrahm\\ \midrule[0.1mm]
1/2 TL& Salz (gestr.)\\ \midrule[0.1mm]
1 TL& 	 Currypulver(gestr.)\\ \midrule[0.1mm]
 etwas&	 Kreuzkümmel, evtl.\\ \midrule[0.1mm]
2 EL&	 Sesam, evtl.\\ \midrule[0.1mm]
 etwas&	 Fett für die Form\\ \midrule[0.1mm]
 evtl.&	 Käse\\ \midrule[0.1mm]
 evtl.&	 Tomate(n)\\ \midrule[0.1mm]
 evtl.&	 Erbsen\\ \midrule[0.1mm]
 evtl.&	 Zucchini\\ \midrule[0.1mm]
 evtl.&	 Kochschinken\\ \bottomrule
\end{tabular}
\end{small}

\end{minipage}
\hfill
\begin{minipage}[t]{0.58\textwidth}
\vspace{0pt}
\subsection*{Zubereitung}
\begin{enumerate}[leftmargin=*, itemindent=14pt]
\item Den Backofen auf 180 °C vorheizen. Die Karotten waschen und schälen. Dünnere längs halbieren, Dickere vierteln. Vom Brokkoli den Strunk abschneiden.\\ 

\item Die Eier mit Sauerrahm verquirlen und würzen. 1,5 EL Sesam dazugeben. Die Form (Quicheform 24 cm) fetten, die Eiermasse einfüllen. Das Gemüse darauf verteilen. Mit Sesam bestreuen und im Backofen (Mitte, Umluft 160 °C) in 35 Minuten stocken lassen. \\

\item Die Frittata in Stücke schneiden und anrichten. Als Beilage passen Brot oder Kartoffeln. Reste kann man auch kalt für die Jause verwenden. \\
\end{enumerate}
Varianten: Gemüse nach Saison variieren, z. B.: 150 g Erbsen und 200 g Cocktailtomaten, 1 Paprikaschote und 2 Zucchini (ca. 160 g), Mozzarella und Tomaten, geraspeltem Käse, 2 Scheiben Kochschinken würfeln und unter die Eiermasse mischen.
\end{minipage}
\vfill
\decothreeright \, \textbf{Arbeitszeit:} ca. 20 Min.	 / \textbf{Schwierigkeitsgrad:} simpel	 / \decothreeleft \hfill Bewertung: \LEFTcircle  \Circle \Circle  \Circle \Circle
\ifoot{\href{http://www.chefkoch.de/rezepte/376181123622498/Ungewoehnlicher-Rote-Bete-Salat.html}{http://www.chefkoch.de/rezepte/376181123622498/Ungewoehnlicher-Rote-Bete-Salat.html}}
\section[Rote-Bete-Salat mit Erdnussbutter]{\leafright\, Rote-Bete-Salat mit Erdnussbutter \,\leafleft}
\begin{minipage}[t]{0.34\textwidth}
\vspace{0pt}
\fbox{\includegraphics[width=1\linewidth]{./Bilder/rote-bete-salat.png}}
\vspace{0.5cm}

\begin{small}
\begin{tabular}{R{1.6cm} L{3.8cm} }
\multicolumn{2}{c}{\textbf{Zutaten für 4 Portionen }}\\ \toprule
4 Knollen&	 Rote Bete (oder ein Glas)\\ \midrule[0.1mm]
2 Zehen&	 Knoblauch\\ \midrule[0.1mm]
2 Becher&	 Joghurt, klein\\ \midrule[0.1mm]
2 EL&	 Erdnussbutter\\ \midrule[0.1mm]
 	& Pfeffer, schwarz, frisch\\ \bottomrule
\end{tabular}
\end{small}
\end{minipage}
\hfill
\begin{minipage}[t]{0.58\textwidth}
\vspace{0pt}
\subsection*{Zubereitung}
\begin{enumerate}[leftmargin=*, itemindent=14pt]
\item Den Joghurt mit der Erdnussbutter, dem durchgepressten Knoblauch und frisch gemahlenem schwarzen Pfeffer mischen, über die in Scheiben geschnittene Rote Bete geben.
\end{enumerate}
\end{minipage}
\vfill
\decothreeright \, \textbf{Arbeitszeit:} 10 Min. / \textbf{Schwierigkeitsgrad:} simpel \decothreeleft \hfill Bewertung:  \Circle \Circle \Circle \Circle \Circle
\ifoot{\href{http://www.chefkoch.de/rezepte/1692841277617126/Brokkoli-Wellness-Salat.html}{http://www.chefkoch.de/rezepte/1692841277617126/Brokkoli-Wellness-Salat.html}}
\section[Brokkoli-Salat]{\leafright\, Brokkoli-Salat \,\leafleft}
\begin{minipage}[t]{0.34\textwidth}
\vspace{0pt}
\fbox{\includegraphics[width=1\linewidth]{./Bilder/brokkoli-salat.png}}
\vspace{0.5cm}

\begin{small}
\begin{tabular}{R{1.6cm} L{3.8cm} }
\multicolumn{2}{c}{\textbf{Zutaten für 4 Portionen }}\\ \toprule
600 g&	 Brokkoli\\ \midrule[0.1mm]
2 &	 Paprikaschote(n), rot\\ \midrule[0.1mm]
2 &	 Apfel\\ \midrule[0.1mm]
60 g&	 Öl\\ \midrule[0.1mm]
30 g&	 Essig\\ \midrule[0.1mm]
2 TL&	 Honig\\ \midrule[0.1mm]
 	& Kräutersalz\\ \midrule[0.1mm]
4 EL&	 Sonnenblumenkerne\\ \bottomrule
\end{tabular}
\end{small}
\end{minipage}
\hfill
\begin{minipage}[t]{0.58\textwidth}
\vspace{0pt}
\subsection*{Zubereitung}
\begin{enumerate}[leftmargin=*, itemindent=14pt]
\item Den Brokkoli in Röschen schneiden und waschen. Die Paprika und den Apfel waschen und klein schneiden.

\item Aus den restlichen Zutaten ein Dressing anrühren und mit Brokkoli, Paprika und Apfel vermischen. Mit den Sonnenblumenkernen bestreut servieren.
\end{enumerate}
\end{minipage}
\vfill
\decothreeright \, \textbf{Arbeitszeit:} 10 Min. / \textbf{Schwierigkeitsgrad:} simpel \decothreeleft \hfill Bewertung:  \Circle \Circle \Circle \Circle \Circle

\chapter{Grillrezepte}
\ifoot{\href{http://goli.magix.net/alle-alben/!/oa/2774343-24054283/}{http://goli.magix.net/alle-alben/!/oa/2774343-24054283/}}
\section[SFB - Baguette]{\leafright\, SFB - Baguette \leafleft}
\begin{minipage}[t]{0.34\textwidth}
\vspace{0pt}
\fbox{\includegraphics[width=1\linewidth]{./Bilder/sfb_brot.png}}
\vspace{0.5cm}

\begin{small}
\begin{tabular}{R{1.6cm} L{3.8cm} }
\multicolumn{2}{c}{\textbf{Zutaten für 4 Portionen }}\\ \toprule
\multicolumn{2}{c}{- Vorteig -}\\ \midrule[0.1mm]
100 ml& warmes Wasser\\ \midrule[0.1mm]
100 ml& warme Milch\\ \midrule[0.1mm]
40 g& Hefe\\ \midrule[0.1mm]
200 g& Weizenmehl glatt W700 oder W480\\ \midrule[0.1mm]
\multicolumn{2}{c}{- Hauptteig -}\\ \midrule[0.1mm]
200 ml& warmes Wasser\\ \midrule[0.1mm]
200 ml& warme Milch\\ \midrule[0.1mm]
800 g& Weizenmehl glatt W700 oder W480\\ \midrule[0.1mm]
20 g& Salz\\ \midrule[0.1mm]
20 g& Öl\\ \midrule[0.1mm]
25 g& Gersten-Malz\\ \bottomrule
\end{tabular}
\end{small}
\end{minipage}
\hfill
\begin{minipage}[t]{0.58\textwidth}
\vspace{0pt}
\subsection*{Zubereitung}
\begin{enumerate}[leftmargin=*, itemindent=14pt]
\item Die Zutaten für den Vorteig verrühren und 10 min zugedeckt an einem warmen Ort ruhen lassen.
\item Die Zutaten für den Hauptteig mit dem Vorteig verkneten und nochmal 10 min an einem warmen Ort zugedeckt ruhen lassen.
\item In 6 gleich große Stücke teilen und in Kugeln formen. Die Kugeln wieder 10 min zugedeckt ruhen lassen.
\item Baguettes formen, auf ein Blech (am besten Lochblech) legen, einschneiden und ca. 30 min zugedeckt aufgehen lassen. Backzeit ca. 20 min bei 180 - 200 °C.
\end{enumerate}
\end{minipage}
\vfill
\decothreeright \, \textbf{Arbeitszeit:} ca. 30 Min. / \textbf{Schwierigkeitsgrad:} simpel \decothreeleft \hfill Bewertung: \Circle  \Circle \Circle  \Circle \Circle
\ifoot{\href{http://goli.magix.net/alle-alben/!/oa/2774343-24325056/}{http://goli.magix.net/alle-alben/!/oa/2774343-24325056/}}
\section[Mais Tortillas]{\leafright\, Mais Tortillas \leafleft}
\begin{minipage}[t]{0.34\textwidth}
\vspace{0pt}
\fbox{\includegraphics[width=1\linewidth]{./Bilder/mais_tortillas.png}}
\vspace{0.5cm}

\begin{small}
\begin{tabular}{R{1.7cm} L{3.8cm} }
\multicolumn{2}{c}{\textbf{Zutaten für 2 Portionen }}\\ \toprule
70 g& Maismehl\\ \midrule[0.1mm]
90 g& Weizenmehl glatt\\ \midrule[0.1mm]
100 - 120 g&Wasser\\ \midrule[0.1mm]
etwas&Salz\\ \bottomrule
\end{tabular}
\end{small}

\end{minipage}
\hfill
\begin{minipage}[t]{0.58\textwidth}
\vspace{0pt}
\subsection*{Zubereitung}
\begin{enumerate}[leftmargin=*, itemindent=14pt]
\item Aus allen Zutaten einen Teig bereiten. 1/2 Stunde in Folie eingewickelt ruhen lassen.
\item 6 gleich große Kugeln formen. Mit dem Nudelholz Fladen möglichst dünn ausrollen.
\item Auf beiden Seiten bei mittlerer Hitze ca. 1 min direkt grillen.
\end{enumerate}
\end{minipage}
\vfill
\decothreeright \, \textbf{Arbeitszeit:} ca. 15 Min. / \textbf{Schwierigkeitsgrad:} simpel \decothreeleft \hfill Bewertung: \CIRCLE \CIRCLE \CIRCLE \CIRCLE  \Circle

\chapter{Fermentiertes}
\ifoot{\href{https://www.kochbar.de/rezept/265348/Beilage-Sauerkraut-selbermachen.html/}{https://www.kochbar.de/rezept/265348/Beilage-Sauerkraut-selbermachen.html/}}
\section[Sauerkraut]{\leafright\, Sauerkraut \leafleft}
\begin{minipage}[t]{0.34\textwidth}
\vspace{0pt}
\fbox{\includegraphics[width=1\linewidth]{./Bilder/sauerkraut-selbermachen-rezept.png}}
\vspace{0.5cm}

\begin{small}
\begin{tabular}{R{1.6cm} L{3.8cm} }
\multicolumn{2}{c}{\textbf{Zutaten für 10 Liter }}\\ \toprule
8kg & Weißkraut / Kohlkopf\\ \midrule[0.1mm]
120g& Salz (15 g pro kg Kraut\\ \bottomrule
\end{tabular}
\end{small}
\end{minipage}
\hfill
\begin{minipage}[t]{0.58\textwidth}
\vspace{0pt}
\subsection*{Zubereitung}
\begin{enumerate}[leftmargin=*, itemindent=14pt]


\item Tontopf mit heißem Wasser ausbrühen und dann an der Luft trocknen. Wenn der Tontopf trocken ist, dann legen Sie ihn mit sauberen Weißkohl-Blättern aus. Dadurch bleibt das Sauerkraut von der Farbe her schön hell. Anschließend beginnen Sie nun mit dem eigentlichen Sauerkraut machen.

\item Zuerst entfernen Sie die äußeren grünen Blätter schneiden den Kohlkopf in der Mitte durch und entfernen den dicken Strunk.

\item Nun geht es mit dem Krauthobel weiter. In einer großen Schüssel oder Kunstoffbadewanne sammeln Sie das geraspelte Kraut. Dieses durchmengen Sie im nächsten Schritt mit Salz, das ist wichtig um zu Beginn der Gärung unerwünschte Mikroben fernzuhalten. Ca. 15 Gramm Salz auf 1kg Kohl.

\item Dieses Gemenge kräftig mit den Händen quetschen oder mit den Fäusten stampfen bis sich ordentlich Brühe bildet. Dann in den Tontopf, ca. ¾ voll, einfüllen. Festdrücken, bis ca. 2 cm Brühe über dem Kraut stehen. Nun werden einige Krautblätter als Abschluß aufgelegt.

\item Danach werden auf das Sauerkraut die Abschlußsteine gelegt und eventuell noch mit einem faustgroßen Kieselstein beschwert. Dies ist nötig, damit das Sauerkraut immer in der Lake bleibt und nicht aufquellen kann. Der Tontopf hat einen zusätzlichen Rand, in dem der Deckel liegt. Dort wird Wasser eingefüllt. So können von außen keine Keime eindringen. Die Gase können darüber leicht entweichen

\item 14 Tage bis 3 Wochen bleibt der Topf nun in einem etwas wärmeren Raum stehen, dann erst kommt er in den Keller. Nach weiteren 1 – 2 Wochen kann man ja mal probieren, ob es schon genügend gesäuert hat.

\item TIPP: In die Tontopfrinne eine starke Salzlake füllen. Es können sich ansonsten in der Rinne Schleimbakterien bilden. Wenn der Topf in einen kühleren Raum gebracht wird, erfolgt beim Temperaturangleich zeitweise ein Sog nach innen. Somit könnten Schleimbakterien ins Kraut gelangen. Meinen ersten Ansatz dieses Jahr konnte ich vermutlich aus diesem Grund entsorgen.
    
\end{enumerate}
\end{minipage}
\vfill
\decothreeright \, \textbf{Arbeitszeit:} - / \textbf{Schwierigkeitsgrad:} leicht \decothreeleft \hfill Bewertung: \Circle  \Circle \Circle  \Circle \Circle


\chapter{Hinweise}
\ifoot{\href{https://missboulette.wordpress.com/2010/07/26/4-schritte-zum-ideal-gerosteten-sesam/}{https://missboulette.wordpress.com/2010/07/26/4-schritte-zum-ideal-gerosteten-sesam/}}
\section[4 Schritte zum ideal gerösteten Sesam]{\leafright\, 4 Schritte zum ideal gerösteten Sesam \leafleft}
\label{sec:4 Schritte zum ideal gerösteten Sesam}

\begin{minipage}[t]{0.34\textwidth}
\vspace{0pt}\fbox{\includegraphics[width=1\linewidth]{./Bilder/ideal_geroesteter_Sesam.png}}
\vspace{0.5cm}
\end{minipage}
\hfill
\begin{minipage}[t]{0.58\textwidth}
\vspace{0pt}
\subsection*{Zubereitung}
\begin{enumerate}[leftmargin=*, itemindent=14pt]

\item Dafür nehme ich 1-2 Tassen Sesam als Portion, kurz waschen, auf einem Haarsieb (falls nicht vorhanden ein Tuch zur Hilfe nehmen) gut abtropfen lassen.

\item Den nassen Sesam in einem Topf auf mittlerer Hitze unter ständigem Rühren mit einem breiten Spatel ohne Fett langsam rösten. Erst verdunstet das Wasser und es entsteht Dampf, aber dann erhitzen sich die Körner langsam und es fängt an zu knistern und zu knacken. Ständig weiter rühren nicht vergessen!! Nach einigen Minuten riecht es schon langsam nussig und aromatisch, die vormals hellen und platten Körner sind nun golden und bauchig, und es raucht sehr dezent. Ab diesem Zeitpunkt ist Multitasking angesagt!

\item Während man mit der einen Hand wie eine Maschine gleichmäßig weiterrührt, probiert man mit der anderen immer wieder und zerreibt dabei einige Sesamkörner zwischen Daumen und Zeigefinger. Erst wenn sich die Körner gut zerreiben lassen und die Finger dabei trocken bleiben, haben sie die richtige Röststufe und damit das volle Aroma erreicht.

\item Nun sofort in eine vorher bereitgestellte Schale (mise en place) umschütten und abkühlen lassen. Es geht hier um Sekunden! Falls ihr erst jetzt eine Schüssel suchen oder umständlich aus dem hintersten Regal herausholen müsstet, wäre es der Tod für die Sesamkörner. Noch schlimmer wäre es für sie im noch heißen Topf zu verweilen. Die ideale Röststufe liegt leider haarscharf neben der ranzigen. Sobald der Zenit erreicht ist fällt die Qualität rasant ab. Erkennen könnt ihr es an übermäßiger Rauchentwicklung, d.h. Fett tritt bereits aus den Körnern aus und verbrennt. Die Körner sind ölig, einige bereits geplatzt. Diese Körner würden nur wenige Tage gut schmecken, danach schnell unangenehm ranzig.
\end{enumerate}

Verpasst ihr den idealen Zeitpunkt dagegen nicht, kann der so geröstete Sesam fast einige Monate gut überstehen. Trocken, luftdicht und dunkel lagern.


Nach Belieben mit Currypulver bestäubt servieren.
\end{minipage}
\vfill
\decothreeright \, \textbf{Arbeitszeit:} wenige Minuten	 / \textbf{Schwierigkeitsgrad:} simpel	 / \decothreeleft \hfill Bewertung: \Circle  \Circle \Circle  \Circle \Circle
\ifoot{\href{http://www.cookingforengineers.com/recipe/44/Basic-Vinaigrette-Salad-Dressing}{http://www.cookingforengineers.com/recipe/44/Basic-Vinaigrette-Salad-Dressing}}
\section[Basic Vinaigrette Salat Dressing]{\leafright\, Basic Vinaigrette Salat Dressing \leafleft}
\begin{minipage}[t]{0.34\textwidth}
\vspace{0pt} %\fbox{\includegraphics[width=1\linewidth]{./Bilder/rezepte_gefuellte-paprika.png}}
\vspace{0.5cm}

\begin{small}
\begin{tabular}{R{1.7 cm} L{3.7cm} }
\multicolumn{2}{c}{\textbf{Zutaten für 1 Portionen}}\\ \toprule
1  Tasse& extra natives Olivenöl\\ \midrule[0.1mm]
1/2 Becher & Balsamico Essig\\ \midrule[0.1mm]
3 & Knoblauchzehen (gepresst)\\ \midrule[0.1mm]
1 Tl. & Oregano\\ \midrule[0.1mm]
1/4 Tl. & Rosmarin, zerdrückt\\ \bottomrule
\end{tabular}
\end{small}
\end{minipage}
\hfill
\begin{minipage}[t]{0.58\textwidth}
\vspace{0pt}
\section*{Zubereitung}
\begin{enumerate}[leftmargin=*, itemindent=14pt]
\item Aus Hackfleisch, Reis, Ei, Zwiebel und Knoblauch einen Hackfleischteig herstellen und mit den Gewürzen abschmecken. In die Paprikaschoten füllen und aus dem Rest Hackfleischbällchen formen.
\item Die Butter in einem Topf schmelzen, das Mehl dazugeben und etwas anrösten. Mit Gemüsebrühe ablöschen, das Tomatenmark dazugeben und aufkochen lassen. Mit Salz, Pfeffer und etwas Zucker abschmecken. 
\item Die gefüllten Paprikaschoten und die Bällchen in die Soße geben und entweder auf dem Herd oder im Backofen 30-40 min schmoren lassen.
\item Dazu gibt es Reis. 
\end{enumerate}
Alternativ kann man auch Tomatenpüree statt des Tomatenmarks verwenden. Dann etwas weniger Brühe verwenden.
\end{minipage}
\vfill
\decothreeright \, \textbf{Arbeitszeit:} ca. 40 Min.	 / \textbf{Schwierigkeitsgrad:} normal	 / \decothreeleft \hfill Bewertung: \Circle  \Circle \Circle  \Circle \Circle

\end{document}
